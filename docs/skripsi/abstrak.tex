\chapter*{\textbf{ABSTRAK}}

\textbf{MOCHAMMAD HANIF RAMADHAN}. Implementasi Modul Indexing Pada Search 
Engine Telusuri Dengan Integrasi Inverted Index Dan Generalized Suffix Tree 
Untuk Mereduksi Waktu Pencarian. Skripsi. Program Studi Ilmu Komputer. Fakultas
Matematika dan Ilmu Pengetahuan Alam, Universitas Negeri Jakarta. Agustus 2023.
Di bawah bimbingan Muhammad Eka Suryana, M.Kom dan Med Irzal, M.Kom.

\vspace{5mm}
\noindent{}
Mesin pencari atau \emph{search engine} adalah program komputer yang digunakan 
untuk melakukan pencarian situs web. Pencarian dapat dilakukan dengan 
mengumpulkan informasi tentang halaman web terlebih dahulu. Karena ukuran data 
yang besar sebagai sumber informasi utama bagi mesin pencari, penggunaan
\textit{index} bisa dimanfaatkan untuk mereduksi waktu pencarian.  Penelitian
ini merupakan bagian dari rangkaian penelitian mesin pencari \textit{Telusuri},
dan bertujuan untuk membuat implementasi \textit{index} berdasarkan struktur
\textit{inverted index}, yaitu struktur \textit{index} yang digunakan oleh mesin
pencari \textit{Google}, dan melakukan integrasi dengan struktur
\textit{Generalized Suffix Tree}. Tujuan utamanya adalah untuk mereduksi waktu
pencarian suatu \textit{keyword} dan memberikan peringkat terhadap dokumen
berdasarkan kesesuaian antara informasi dalam dokumen dengan teks yang
dimasukkan oleh pengguna. Hasil akhir dari implementasi modul menunjukkan
reduksi waktu yang signifikan, berkisar antara 89 - 98\%.

\vspace{5mm}
\noindent{}
\textbf{Kata kunci}: \textit{mesin pencari, indeks, basis data, teori informasi}
