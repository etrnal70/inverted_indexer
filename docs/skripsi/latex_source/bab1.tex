%!TEX root = ./template-skripsi.tex
%-------------------------------------------------------------------------------
% 								BAB I
% 							LATAR BELAKANG
%-------------------------------------------------------------------------------

\chapter{PENDAHULUAN}

\section{Latar Belakang Masalah}

% Intro
Mesin pencari atau \emph{search engine} adalah program komputer yang digunakan 
untuk melakukan pencarian situs web. Pada awal kemunculannya, mesin pencari
lebih diperuntukkan bagi kalangan akademisi untuk mencari jurnal atau dokumen
akademik lainnya. Namun, seiring dengan meluasnya adopsi internet ke berbagai
lapisan masyarakat, mesin pencari memiliki peran yang lebih besar dalam
penggunaan internet. Mesin pencari berubah menjadi sarana utama dalam pencarian
informasi secara umum.

% Sejarah mesin pencari
Bila dilihat dari penggunaannya, mesin pencari bekerja dengan cara mengolah
\emph{query} yang diberikan oleh pengguna dan menampilkan hasil terkait
informasi tertentu di internet yang paling sesuai dengan \emph{query} tersebut.
Tetapi mesin pencari tidak dapat langsung mencari halaman secara langsung 
melalui internet. Mesin pencari membutuhkan database yang berisi informasi pada
halaman yang tersebar di seluruh jaringan internet. 

Ketika internet masih memiliki lingkup pengguna yang kecil, metode pengumpulan
data untuk database mesin pencari masih terbilang primitif. Beberapa orang 
membuat katalog dari berbagai situs dengan kategori tertentu secara manual dan 
memperbaruinya dari waktu ke waktu (\cite{seymour2011history}). Namun, makin
tingginya adopsi internet membuat jumlah data pada internet meledak. Hal ini
berakibat pada diperlukannya metode pengumpulan data yang lebih efisien dan
metode penerimaan informasi yang lebih cepat.

% Arsitektur Google
Mesin pencari membutuhkan serangkaian proses yang perlu dilakukan sebelum dapat
menerima \emph{query} terkait informasi tertentu. \emph{Google}, mesin pencari
paling populer saat ini, mempunyai beberapa komponen yang menjalankan tugas
secara spesifik. Arsitektur \emph{Google} terdiri dari beberapa komponen utama
seperti \emph{crawler}, modul \emph{indexer}, modul pemeringkatan
\emph{PageRank} dan modul pencari (\cite{brin1998google}). Seluruh modul
tersebut dibuat secara mandiri tanpa menggunakan aplikasi atau layanan pihak
ketiga.

Meski dengan informasi tentang rancangan arsitektur dari publikasi artikel,
implementasi ulang arsitektur \emph{Google} secara menyeluruh cukup sulit
dilakukan karena banyak detail dari arsitektur yang tidak dicantumkan. Selain
itu \emph{Google} juga banyak membuat implementasi yang telah dioptimalkan untuk
kebutuhan mesin pencari baik dari segi performa maupun penggunaan ruang
penyimpanannya, tanpa memublikasikan kode program atau algoritma yang digunakan.

% Arsitektur Telusuri
Upaya implementasi ulang arsitektur \emph{Google} telah dilakukan
(\cite{lazuardy2023search}) dan menghasilkan integrasi mesin pencari secara
utuh. Arsitektur tersebut merupakan pengembangan dari penelitian yang telah
dilakukan sebelumnya yang menghasilkan \emph{web crawler}
(\cite{fathan2021crawler}). \emph{Web crawler} bertugas mengumpulkan halaman web
berdasarkan \emph{entrypoint} tertentu. Halaman web tersebut kemudian di ekstrak
dan dikumpulkan daftar kata yang termuat pada halaman web tersebut serta
\emph{outgoing link} yang merujuk kepada halaman web lain. Kedua data tersebut
kemudian disimpan pada database.

Data yang tersimpan pada database selanjutnya di proses oleh modul
\emph{PageRank}. Modul \emph{PageRank} menghitung peringkat suatu halaman web
berdasarkan seberapa banyak halaman web lain yang merujuk kepada halaman web
tersebut. Setelah kalkulasi peringkat halaman, data akan diolah oleh modul
\emph{TF-IDF}. Modul \emph{TF-IDF} akan menghitung bobot kata pada dokumen
berdasarkan frekuensi kemunculan kata tersebut dalam suatu dokumen tertentu dan
jumlah dokumen yang mengandung kata tersebut. Hasil perhitungan skor dari modul
\emph{PageRank} dan \emph{TF-IDF} kemudian akan digabungkan dengan menggunakan
algoritma \emph{similarity scoring} yang akan menghasilkan skor relevansi suatu
halaman.

% Kekurangan dari arsitektur existing terutama pada bagian indexing
Arsitektur \textit{Telusuri} saat ini masih memiliki berbagai kekurangan, dan
implementasinya juga cukup berbeda dibandingkan dengan arsitektur milik
\emph{Google} karena keterbatasan detailnya. Salah satu modul yang belum
memiliki implementasi yang sesuai adalah modul \emph{indexing}. \emph{Indexing}
adalah suatu proses pemetaan \emph{record} pada database dengan tujuan
mempercepat proses pengambilan \emph{record} dari database dan meningkatkan
relevansi hasil pencarian. Proses \emph{indexing} akan menghasilkan
\emph{index}, yaitu suatu representasi data yang merujuk pada lokasi data yang
lebih lengkap. Konsep penggunaan \emph{index} pada mesin pencari sama dengan
index yang biasa ditemukan di bagian belakang buku.

% Penjelasan terkait indexing
Representasi data pada \emph{index} memiliki tingkat akurasi lokasi data
(\emph{granularity}) yang dapat diatur, seperti suatu frasa dalam suatu
paragraf, atau unit data yang lebih kecil seperti satu kata tertentu saja.
Pemilihan tingkat \emph{granularity} dapat memengaruhi performa dan akurasi
dari proses pengambilan data. Sebagai contoh, penggunaan \emph{synonym ring}
yang dapat melakukan pengelompokan kata yang bermakna sama seperti
\emph{WordNet} dapat memberikan hasil yang lebih baik dibandingkan dengan hanya
menggunakan potongan kata biasa (\cite{gonzalo1998wordnet}).

% Indexing untuk penyimpanan teks
Terdapat berbagai implementasi \emph{index} yang disesuaikan dengan jenis data
yang disimpan pada database. Mesin pencari membutuhkan implementasi \emph{index}
yang dioptimalkan untuk teks secara menyeluruh. Dalam artikel yang berjudul
\emph{An Efficient Indexing Technique for Full-Text Database Systems}, 
implementasi \emph{index} yang difokuskan kepada penyimpanan teks setidaknya
perlu melakukan tiga hal secara efisien. Yang pertama adalah mendukung
pengambilan dokumen berdasarkan \emph{query} berupa beberapa kata yang
digabungkan oleh operator logika. Yang kedua adalah memiliki kemampuan untuk
menambahkan \emph{record} baru secara efisien. Yang ketiga adalah kemampuan
untuk membuat pemeringkatan terhadap \emph{record} yang ada apabila tidak ada
data yang memenuhi \emph{query} secara penuh (\cite{zobel1992efficient}).

% Prinsip kerja
Dari persyaratan di atas, solusi yang umum digunakan untuk database penyimpanan
teks adalah \emph{inverted file index} atau biasa disebut \emph{inverted index}
saja. Wujud dari \emph{inverted index} adalah sebuah struktur data yang 
memetakan kosakata dengan dokumen atau teks utama tempat kosakata tersebut
berada(\cite{hersh2001gigabytes}). \textit{Google} juga menggunakan struktur 
\textit{inverted index} pada implementasi arsitekturnya (\cite{brin1998google}).

% Pross indexing Google
Proses \emph{indexing} yang akan di replika akan menggunakan detail yang
terbatas yang dapat diakses pada \emph{paper Google}.  Untuk memroses
\emph{query}, \emph{Google} melakukan beberapa langkah berikut.  Pertama,
\emph{query} dipotong per kata. Setiap kata kemudian di konversi menjadi
\emph{wordID}, yang dapat mengidentifikasi suatu kata secara unik.

Dari kumpulan \emph{wordID} tersebut, untuk setiap kata akan dicari
kemunculannya pada permulaan daftar \emph{index} yang berisi judul dan
\emph{outgoing links}.  Pencarian dilakukan dengan proses \emph{scanning} secara
menyeluruh pada daftar \emph{index} sampai ditemukan dokumen yang memenuhi
seluruh kata pada \emph{query}. Untuk setiap dokumen yang ditemukan, akan
dihitung skor dokumen berdasarkan \emph{query} yang ada.

Jika telah sampai pada akhir daftar dokumen dan posisi pencarian berada pada
daftar judul dan \emph{outgoing links}, maka posisi akan berpindah ke daftar
kata pada seluruh dokumen, dan mulai melakukan pencarian dokumen lagi. Tetapi
jika belum mencapai akhir daftar dokumen, maka posisi pencarian akan diulang
dari posisi awal dan memulai pencarian lagi. Setelah seluruh daftar
\emph{index} dikunjungi, nantinya akan diakhiri dengan mengurutkan dokumen
berdasarkan skor.

Penelitian tentang struktur data untuk keperluan \emph{indexing} telah dilakukan
dan menghasilkan implementasi modul \emph{indexing} dengan menggunakan struktur
\emph{generalized suffix tree} (\emph{GST}) (\cite{zaidan2023gst}). Implementasi
dari metode \emph{inverted index} dapat menggantikan implementasi modul yang 
sudah ada secara menyeluruh atau diintegrasikan dengan modul yang sudah ada 
sebagai bentuk peningkatan kemampuan modul. Modul \emph{indexing} nantinya juga
dapat berperan sebagai metode \emph{query ranking} yang berbeda, atau melakukan
enkapsulasi nilai \emph{query ranking} yang sudah ada.

\section{Rumusan Masalah}
Berdasarkan uraian pada latar belakang yang di utarakan di atas, maka perumusan
masalah pada penelitian ini adalah "\textbf{bagaimana meningkatkan performa 
waktu pencarian dengan menggunakan \textit{index} ?}".

\section{Batasan Masalah}

Batasan masalah pada penelitian ini adalah pembuatan salah satu komponen 
dalam arsitektur mesin pencari yaitu modul \emph{indexing}. Sistem
\emph{indexing} yang akan dibuat mengacu pada konsep arsitektur \emph{Google}
(\cite{brin1998google}). Selain itu, implementasi sistem \textit{indexing} ini
hanya akan mencakup proses konstruksi awal index berdasarkan data yang sudah
dikumpulkan oleh \textit{crawler}.

\section{Tujuan Penelitian}
\begin{enumerate}
	\item{Membuat implementasi \emph{indexing} berdasarkan konsep \textit{inverted
		index} untuk memenuhi kebutuhan mesin pencari}
	\item{Melakukan integrasi antara struktur \textit{inverted index} dengan
		\textit{Generalized Suffix Tree} yang sudah dibuat pada penelitian 
		sebelumnya}
\end{enumerate}

\section{Manfaat Penelitian}
\begin{enumerate}
	\item Bagi penulis
		
	Menambah pengetahuan dibidang \emph{information retrieval} khususnya mengenai
		\emph{search engine} dan \emph{crawling}, mengasah kemampuan
		\emph{programming}, dan memperoleh gelar sarjana dibidang Ilmu Komputer.
		Selain itu, penulisan ini juga merupakan media bagi penulis untuk
		mengaplikasikan ilmu yang didapat di kampus ke kehidupan masyarakat.
		
	\item Bagi Universitas Negeri Jakarta
	
	Menjadi pertimbangan dan evaluasi akademik khususnya Program Studi Ilmu
		Komputer dalam penyusunan skripsi sehingga dapat meningkatkan kualitas
		akademik di program studi Ilmu Komputer Universitas Negeri Jakarta serta
		meningkatkan kualitas lulusannya.
			
\end{enumerate}

% Baris ini digunakan untuk membantu dalam melakukan sitasi
% Karena diapit dengan comment, maka baris ini akan diabaikan
% oleh compiler LaTeX.
\begin{comment}
\bibliography{daftar-pustaka}
\end{comment}
