\chapter*{\textbf{ABSTRACT}}

\textbf{MOCHAMMAD HANIF RAMADHAN}. Implementasi Modul Indexing Pada Search 
Engine Telusuri Dengan Integrasi Inverted Index Dan Generalized Suffix Tree 
Untuk Mereduksi Waktu Pencarian. Mini Thesis. Computer Science. Faculty of 
Mathematics and Natural Sciences, Universitas Negeri Jakarta. August 2023.
Under guidance from Muhammad Eka Suryana, M.Kom and Med Irzal, M.Kom.

\vspace{5mm}
\noindent{}
Search engine is a computer program that is used to search for a webpage. Before 
searching can be done, it is required to gather informatoin about the webpage 
first. Due to the large collection of data needed for a search engine to be 
usable, indexes can be used to reduce time to retrieve information. This 
research is a part of a longer research for \textit{Telusuri} search engine, and
aims to implement an index structure based on inverted index form, which are 
used by \textit{Google}, and integrates it with the \textit{Generalized Suffix 
Tree}. The goals is to reduce the time taken to get informations from a given 
keyword and ranks the result based on the relevancy between the webpage and the 
input keyword. The index implementations are shown to be able to reduce the time 
taken significantly, ranging from 89 to 98\%.

\vspace{5mm}
\noindent{}
\textbf{Kata kunci}: \textit{search engine, indexing, database, information 
retrieval, tree}
