%!TEX root = ./template-skripsi.tex
%-------------------------------------------------------------------------------
%                          BAB V
%               		KESIMPULAN DAN SARAN
%-------------------------------------------------------------------------------

\chapter{KESIMPULAN DAN SARAN}

\section{Kesimpulan}

Berdasarkan hasil dari implementasi dan pengujian terhadap arsitektur
\textit{inverted index}, maka diperoleh kesimpulan sebagai berikut:

\begin{enumerate}
	\item{Data pada kolom \textit{paragraph} dengan jumlah 1.010.083 baris dari
		tabel \textit{page\_paragraph} berhasil digunakan untuk membuat struktur
		\textit{inverted index}.}
	\item{Modul \textit{indexer} berbasis \textit{inverted index} yang digunakan
		untuk melakukan pemeringkatan pada \textit{query} yang diberikan oleh 
		pengguna telah selesai dibuat.}
	\item{Penyimpanan persisten untuk struktur \textit{inverted index} 
		diimplementasikan sebagai penyimpanan sederhana dalam satu proses yang sama 
		dengan modul \textit{indexer}.}
	\item{Integrasi dengan struktur \textit{GST} yang telah dibuat pada penelitian 
		sebelumnya berhasil diintegrasikan dengan struktur \textit{inverted index}.}
	\item{Integrasi dengan struktur \textit{GST} dapat memberikan reduksi waktu 
		pencarian yang signifikan jika dibandingkan hanya dengan penggunaan
		\textit{inverted index} saja, dengan rasio perbedaan berkisar antara 85 
		- 98\%.}
	\item{Relevansi pencarian tidak terlalu baik jika hanya menggunakan
		\textit{inverted index}. Hasil menjadi lebih relevan ketika menggunakan 
		integrasi \textit{GST}.}
	\item{Informasi yang ada pada \textit{tag} paragraf (\textit{<p>}) tidak cukup 
		untuk menjadi dasar dalam pemeringkatan hasil pencarian.}
\end{enumerate}

\section{Saran}

Adapun saran untuk penelitian selanjutnya adalah:
\begin{enumerate} 
	\item{Melanjutkan penelitian tentang informasi lain yang dapat di ekstrak dari 
		suatu halaman web untuk meningkatkan hasil pencarian.}
	\item{Melakukan perombakan terhadap struktur \textit{hitlists} agar bisa 
		direferensikan melalui alamat memori untuk menghilangkan redundansi pada 
		struktur dengan integrasi \textit{GST}.}
	\item{Melanjutkan penelitian tentang penggunaan kompresi pada struktur
		\textit{hit} untuk mereduksi penggunaan memori.}
	\item{Menuliskan implementasi \textit{indexer} dengan menggunakan bahasa lain 
		yang lebih cepat dan mampu mengeksploitasi sistem \textit{multi-thread} 
		dengan lebih baik seperti \textit{Rust} atau \textit{C++}.}
	\item{Melanjutkan penelitian tentang distribusi penyimpanan dan penggunaan 
		\textit{hitlists} pada memori.}
\end{enumerate}


% Baris ini digunakan untuk membantu dalam melakukan sitasi
% Karena diapit dengan comment, maka baris ini akan diabaikan
% oleh compiler LaTeX.
\begin{comment}
\bibliography{daftar-pustaka}
\end{comment}
