%-------------------------------------------------------------------------------
%                      Template Naskah Skripsi
%               	Berdasarkan format JTETI FT UGM
% 						(c) @gunturdputra 2014
%-------------------------------------------------------------------------------

%Template pembuatan naskah skripsi.
\documentclass{jtetiskripsi}

%Untuk prefiks pada daftar gambar dan tabel
\usepackage[titles]{tocloft}
\renewcommand\cftfigpresnum{Gambar\  }
\renewcommand\cfttabpresnum{Tabel\   }

%Untuk hyperlink dan table of content
\usepackage[hidelinks]{hyperref}
\newlength{\mylenf}
\settowidth{\mylenf}{\cftfigpresnum}
\setlength{\cftfignumwidth}{\dimexpr\mylenf+2em}
\setlength{\cfttabnumwidth}{\dimexpr\mylenf+2em}

%Untuk Bold Face pada Keterangan Gambar
\usepackage[labelfont=bf]{caption}

%Untuk caption dan subcaption
\usepackage{caption}
\usepackage{subcaption}

%pdf
\usepackage{pdfpages}

%table
\usepackage{graphics}

\usepackage{wrapfig}

%bibliography
\usepackage[numbers]{natbib}

%equation
\usepackage{amsmath}

%-----------------------------------------------------------------
%Disini awal masukan untuk data proposal skripsi
%-----------------------------------------------------------------
\titleind{Perancangan Indexing Luar Biasa Bismillah 4 Tahun Aamiinn}

\fullname{Mochammad Hanif Ramadhan}

\idnum{3145161299}

%\approvaldate{12 Februari 2019}
\approvaldate{12 Februari 2019}

\degree{Sarjana Ilmu Komputer}

\yearsubmit{2022}

\program{Ilmu Komputer}

\dept{Ilmu Komputer}

\firstsupervisor{Med Irzal, M. Kom.}
\firstnip{197706152003121001}

\secondsupervisor{Muhammad Eka Suryana, M. Kom.}
\secondnip{198512232012121002}

%-----------------------------------------------------------------
%Disini akhir masukan untuk data proposal skripsi
%-----------------------------------------------------------------

\tolerance=1
\emergencystretch=\maxdimen
\hyphenpenalty=10000
\hbadness=10000

\begin{document}

\cover
%-----------------------------------------------------------------

%-----------------------------------------------------------------
%Disini akhir masukan untuk muka skripsi
%-----------------------------------------------------------------

\tableofcontents 
\addcontentsline{toc}{chapter}{DAFTAR ISI}
\listoffigures
\addcontentsline{toc}{chapter}{DAFTAR GAMBAR}
\listoftables
\addcontentsline{toc}{chapter}{DAFTAR TABEL}

\begin{counterpage}
\end{counterpage}
%Disini awal masukan untuk Bab
%-----------------------------------------------------------------
%!TEX root = ./template-skripsi.tex
%-------------------------------------------------------------------------------
% 								BAB I
% 							LATAR BELAKANG
%-------------------------------------------------------------------------------

\chapter{PENDAHULUAN}

\section{Latar Belakang Masalah }

% Intro
Mesin pencari atau \emph{search engine} adalah program komputer yang digunakan 
untuk melakukan pencarian situs web. Mesin pencari membutuhkan data ketersediaan
situs web sebelum dapat digunakan.

% Sejarah mesin pencari
Dalam perkembangannya, berbagai metode digunakan untuk mengumpulkan dan
menyimpan data situs web yang tersedia. \emph{Archie}, mesin pencari pertama, 
mengumpulkan data dengan cara mencari daftar situs yang tersedia secara publik
menggunakan protokol \emph{Telnet} dan memperbaharui nya secara berkala.

% Arsitektur Google
Mesin pencari membutuhkan serangkaian proses yang perlu dilakukan sebelum dapat
menerima \emph{query} terkait informasi tertentu. Pada arsitektur \emph{Google},
terdapat beberapa komponen yang memiliki tugasnya tersendiri. Pertama,
diperlukan proses \emph{crawling} untuk mendapatkan kumpulan \emph{web page}.
Proses ini dilakukan oleh beberapa \emph{crawler}.

% Arsitektur Lazuardy
Upaya implementasi ulang arsitektur Google telah dilakukan oleh Lazuardy
Khatulistiwa dari Ilmu Komputer angkatan 18 Universitas Negeri Jakarta.
Arsitektur yang ada saat ini dimulai dari \emph{crawler} yang bertugas
mengumpulkan halaman web berdasarkan \emph{entrypoint} tertentu. Halaman web
tersebut kemudian di ekstrak dan dikumpulkan daftar kata yang termuat pada
halaman web tersebut serta \emph{outgoing link} yang merujuk kepada halaman web
lain. Kedua data tersebut kemudian disimpan pada database	\emph{MySQL}.

Data yang tersimpan pada database selanjutnya di proses oleh modul
\emph{PageRank}. Modul \emph{PageRank} menghitung peringkat suatu halaman web
berdasarkan seberapa banyak halaman web lain yang merujuk kepada halaman web
tersebut. Setelah kalkulasi peringkat halaman, data akan diolah oleh modul
\emph{TF-IDF}. Modul \emph{TF-IDF} akan menghitung bobot kata pada dokumen
berdasarkan frekuensi kemunculan kata tersebut dalam suatu dokumen tertentu dan
jumlah dokumen yang mengandung kata tersebut.

% Kekurangan dari arsitektur existing terutama pada bagian indexing
Hasil riset yang telah dilakukan oleh Lazuardy masih memiliki kekurangan, dan
implementasinya juga cukup berbeda dibandingkan dengan arsitektur milik Google.
Terdapat banyak komponen dari arsitektur Google yang belum diketahui detailnya,
salah satunya adalah terkait \emph{indexing}.

% Penjelasan terkait indexing
\emph{Indexing} adalah suatu proses pemetaan \emph{record} pada database dengan
tujuan mempercepat proses pengambilan \emph{record} dari database. Proses
\emph{indexing} akan menghasilkan \emph{index}, yaitu representasi data yang
merujuk pada lokasi data yang lebih lengkap pada database. Konsep \emph{index}
disini hampir sama dengan index yang biasa ditemukan di bagian belakang buku.

Representasi data oleh \emph{index} memiliki tingkat akurasi lokasi data
(\emph{granularity}) yang dapat diatur, seperti suatu frase dalam suatu
paragraf, atau unit data yang lebih kecil seperti satu kata tertentu saja.
Pemilihan tingkat \emph{granularity} dapat mempengaruhi performa dan akurasi
dari proses pengambilan data. Sebagai contoh, penggunaan \emph{synonym ring}
yang dapat melakukan pengelompokan kata yang bermakna sama seperti
\emph{WordNet} dapat memberikan hasil yang lebih baik dibandingkan dengan hanya
menggunakan potongan kata biasa~~\cite{gonzalo1998wordnet}.

% Prinsip kerja
Terdapat berbagai implementasi \emph{index} yang disesuaikan dengan jenis data
yang disimpan pada database. Pada kasus yang melibatkan teks, jenis yang paling
umum digunakan adalah \emph{inverted file index}. \emph{Inverted file index}
adalah daftar yang memuat kata dan posisinya dalam dokumen yang diurutkan
berdasarkan kata.  Untuk mendapatkan daftar kata pada dokumen, dibutuhkan
\emph{lexicon} yang memuat seluruh kata yang muncul pada database
~\cite{hersh2001gigabytes}.

% Google menggunakan algoritma indexing sendiri.
Saat ini implementasi ulang proses \emph{indexing} yang tepat untuk kebutuhan
mesin pencari sulit dilakukan karena detail dari arsitektur mesin pencari yang
bersifat tertutup. Google menggunakan implementasi \emph{filesystem} sendiri
yang sudah dioptimalkan baik performa maupun penggunaan \emph{storage}-nya
untuk kebutuhan mesin pencari.

% Jelaskan bahwa pada penelitian lazuardy baru PageRank dan TFIDF saja, sehingga nilai W nya belum jadi
% (masih ada beberapa metode yang perlu digabungkan untuk mendapatkan nilai dari W)
Arsitektur yang dibuat saat ini nantinya akan menggabungkan beberapa algoritma
pemeringkatan untuk mendapatkan skor \emph{query ranking}, dengan rumus:

\begin{equation}
	QR = \beta{}_1 \cdot{} QR_1 + \cdots{} + \beta{}_N \cdot{} QR_N
\end{equation}


% Tanyakan maksud dari indexing mengkapsulasi QR
% Indexing nantinya tetap akan berpengaruh kepada hasil ranking
% Rumuns pencarian dokumen
% Hasil W sudah ada
% Tutup penjelasan dengan menjelaskan langkah Google hingga sebelum proses
% ranking


% Crawler sangat dibutuhkan \emph{search engine} untuk dapat mencari informasi dengan lebih efisien. Proposal ini juga untuk memperdalam ilmu mengenai \emph{information retrieval} khususnya \emph{web crawling}. Oleh karena itu, perlu dibuat desain perancangan \emph{web crawler} pada \emph{search engine}. Dan ini tertuang pada penelitian yang berjudul \textbf{“Desain Perancangan \emph{Crawler} Sebagai Pendukung Pada \emph{Search Engine}”}.


\section{Rumusan Masalah}
Berdasarkan uraian pada latar belakang yang di utarakan di atas, maka perumusan
masalah pada penelitian ini adalah ``Bagaimana cara mendesain perancangan
\emph{crawler} sebagai pendukung pada \emph{search engine}?``

\section{Pembatasan Masalah}
Pembatasan masalah pada penelitian ini adalah pembuatan sebagian arsitektur
\emph{search engine} yaitu \emph{crawling}. Algoritma crawler yang akan dibuat
mengacu pada model algoritma \emph{Google} awal.

\section{Tujuan Penelitian}
\begin{enumerate}
	\item Memahami arsitektur mesin pencari.
	\item Memahami cara kerja proses \emph{indexing}.
	\item Membuat implementasi \emph{indexing} untuk memenuhi kebutuhan mesin
		pencari.
\end{enumerate}

\section{Manfaat Penelitian}
\begin{enumerate}
	\item Bagi penulis
		
	Menambah pengetahuan dibidang \emph{information retrieval} khususnya mengenai
		\emph{search engine} dan \emph{crawling}, mengasah kemampuan
		\emph{programming}, dan memperoleh gelar sarjana dibidang Ilmu Komputer.
		Selain itu, penulisan ini juga merupakan media bagi penulis untuk
		mengaplikasikan ilmu yang didapat di kampus ke kehidupan masyarakat.
		
	\item Bagi Universitas Negeri Jakarta 
	
	Menjadi pertimbangan dan evaluasi akademik khususnya Program Studi Ilmu
		Komputer dalam penyusunan skripsi sehingga dapat meningkatkan kualitas
		akademik di program studi Ilmu Komputer Universitas Negeri Jakarta serta
		meningkatkan kualitas lulusannya.
			
\end{enumerate}

% Baris ini digunakan untuk membantu dalam melakukan sitasi
% Karena diapit dengan comment, maka baris ini akan diabaikan
% oleh compiler LaTeX.
\begin{comment}
\bibliography{daftar-pustaka}
\end{comment}

%!TEX root = ./template-skripsi.tex
%-------------------------------------------------------------------------------
%                            BAB II
%               KAJIAN TEORI
%-------------------------------------------------------------------------------

\chapter{KAJIAN PUSTAKA}

\section{Proses \emph{Indexing}}

\emph{Indexing} adalah proses pemetaan seluruh data pada \emph{database}.
Proses dimulai dengan mengambil data mentah dari tempat penyimpanan halaman web
yang telah dikumpulkan oleh \emph{crawler} (\textit{repository}). Data tersebut
kemudian akan dipetakan sesuai dengan struktur index yang digunakan. Hasil dari
proses indexing adalah sebuah daftar kosakata dengan pasangan informasi berupa
lokasi, yang memiliki cara kerja yang serupa dengan index yang biasa ditemui
pada bagian belakang sebuah buku. Index adalah suatu representasi data yang
merujuk kepada lokasi data yang lebih lengkap di dalam database.

Index merupakan komponen penting dalam suatu arsitektur mesin pencari.
Penggunaan index memberikan pengaruh signifikan dalam proses penerimaan data
dari database karena dapat mengurangi waktu akses \emph{storage}. Hal ini
sejalan dengan pola penggunaan mesin pencari, dimana mayoritas operasi yang
dilakukan adalah memberikan informasi berdasarkan \emph{query} dari pengguna.
Selain itu, representasi index tertentu juga dapat mempengaruhi pengolahan
\emph{query}, penggunaan ruang pada media penyimpanan dan akurasi dari hasil
\emph{query} tersebut.

% TODO Verif ke pak Eka apakah ini masih perlu atau tidak
% Terkait apakah database yang dibutuhkan adalah untuk full-text
Database yang digunakan oleh mesin pencari adalah database yang dirancang
untuk kebutuhan penyimpanan teks secara penuh. Untuk keperluan tersebut,
setidaknya terdapat tiga syarat yang perlu dipenuhi ketika ingin menentukan
representasi index.

\begin{enumerate}
  \item{Memungkinkan untuk mendapatkan data berdasarkan suatu query}

  Dari suatu query pencarian, database harus dapat memberikan hasil yang dapat
  memenuhi query tersebut. Pada konteks penerimaan data yang berbentuk teks,
  konjungsi query sangat umum digunakan. Dalam kasus ini, index harus mampu
  memberikan informasi apakah suatu \emph{record} mengandung kata tertentu, dan
  secara langsung mempengaruhi penentuan granularitas index.

  \item{Menambahkan \emph{record} baru secara efisien}

  Database yang hanya menyimpan teks umumnya digunakan untuk keperluan
  penyimpanan. Pada kasus mesin pencari, operasi yang sering dilakukan adalah
  penulisan data dari \emph{crawler} ke database dan pembacaan data berdasarkan
  query. Operasi untuk mengubah atau menghapus data jarang diperlukan, walaupun
  operasi tersebut tetap perlu didukung.

  \item{Memberikan skor terhadap hasil dari query yang bersifat `informal'}

  Pada kasus mesin pencari, pengguna seringkali memberikan query yang bersifat
  `informal', yaitu query yang tidak dapat dipenuhi seluruhnya oleh database.
  Dalam hal ini, database perlu memberikan hasil yang paling mendekati
  ekspektasi dari query tersebut dengan cara menyertakan nilai relevansi suatu
  data terhadap query.

  Mesin pencari setidaknya membutuhkan syarat pertama dan kedua untuk dipenuhi
  oleh database agar dapat digunakan di dunia nyata.

\end{enumerate}

\section{\emph{Repository}}

\emph{Repository} merupakan tempat penyimpanan seluruh kode HTML dari setiap
halaman web yang telah di-\emph{crawl}. Setiap dokumen memiliki data tambahan
yang disematkan kepadanya, yaitu nomor \emph{id} yang dibuatkan untuk dokumen
tersebut (\textit{docID}), panjang dari dokumen, dan \emph{URL} dari dokumen
tersebut. Seluruh data tersebut dikompres dengan library \emph{zlib} untuk
mengurangi penggunaan ruang pada \emph{repository}.

\section{Daftar URL}

Ketika mengolah data dari \emph{repository}, terdapat langkah tambahan yang
dilakukan jika menemui \emph{anchor text}. \emph{Anchor text} adalah teks berupa
URL yang merujuk kepada dokumen lain. Setiap \emph{anchor text} yang ditemukan
pada data dari \emph{repository} akan disimpan pada sebuah daftar URL\@.

\section{\emph{Document Index}}
\emph{Document index} menyimpan beberapa informasi tambahan pada setiap dokumen.
Setiap index akan memiliki nomor dokumen, status dokumen, alamat tempat dokumen
tersimpan di \emph{repository}, dan \emph{checksum}. Apabila pada suatu dokumen
telah dilakukan proses \emph{crawling}, maka akan disematkan data tambahan yang
berisi sepasang informasi berupa URL dan judul dari dokumen tersebut. Jika
belum melalui tahap \emph{crawling}, maka data tambahan tersebut hanya merujuk
kepada URL yang terdapat pada daftar URL\@.

\section{Daftar Kosakata}

Setiap hasil pengolahan data dari \emph{repository} akan memberikan daftar
kosakata yang bersifat unik. Daftar kosakata tersebut kemudian akan digabungkan
kedalam daftar kosakata yang menampung keseluruhan kosakata pada database yang
telah melalui proses \emph{indexing}.

Implementasi daftar kosakata ini terbagi menjadi dua bagian, yaitu daftar
kosakata itu sendiri dan daftar \emph{pointer} yang merujuk kepada dokumen
tempat kosakata itu berada. Selain itu, daftar kosakata dapat menyematkan
informasi tambahan seperti jumlah dokumen tempat kosakata tersebut muncul.

Untuk mencegah akses ke ruang penyimpanan yang berlebihan, salah satu syarat
dasar dari implementasi index adalah daftar seluruh kosakata dapat dimuat dalam
memori. Hal ini bertujuan untuk memastikan bahwa dalam situasi pencarian hasil
dengan query boolean, hanya membutuhkan setidaknya satu kali akses ke ruang
penyimpanan per kata.

\section{\textit{docID} dan \textit{wordID}}
Untuk merepresentasikan dokumen dan kosakata secara unik, akan digunakan suatu
format penamaan khusus. Dengan mempertimbangkan kemungkinan terjadinya
\textit{collision} pada \textit{identifier} dari dokumen dan kosakata, maka 
akan digunakan teknik \textit{hashing} untuk memberikan \textit{randomness} yang
lebih baik.

% TODO: Penjelasan kenapa pilih MurmurHash3. Sama paper pendukung kalo bisa
% Jelasin juga panjang maks. Rencananya pake konfigurasi 128bit dengan mesin 64bit
Algoritma \textit{hashing} yang akan digunakan adalah \textit{MurmurHash3}.
Alasan dipilihnya algoritma tersebut karena huhu

% TODO: Kayaknya perlu ditambahin variabel lain
Untuk kosakata, proses \textit{hashing} akan dilakukan terhadap kosakata itu
sendiri. Sementara untuk dokumen, proses \textit{hashing} akan dilakukan terhadap
gabungan dari dua komponen dokumen, yaitu tanggal ketika dokumen tersebut
diunduh dan \textit{URL} dari dokumen itu sendiri.

\section{\emph{Inverted Index}}

\emph{Inverted index} merupakan jenis index yang memetakan potongan isi dari
suatu dokumen atau koleksi dokumen terhadap lokasi aslinya di dokumen atau
koleksi dokumen tersebut. Karena bentuk strukturnya, inverted index cocok
digunakan untuk kebutuhan penerimaan data berdasarkan query seperti mesin
pencari.

\subsection{Struktur Index}

Representasi dari kemunculan suatu kata pada dokumen, atau biasa disebut sebagai
\emph{hit}, memiliki beberapa data tertentu terkait kata yang dikandung. Setiap
\emph{hit} merepresentasikan suatu kemunculan kata beserta data berikut:

\begin{itemize}
  \item{Lokasi kata dalam dokumen}
  \item{Jenis \emph{font} yang digunakan}
  \item{Informasi terkait penggunaan huruf kapital}
\end{itemize}

\emph{Hit} terbagi menjadi tiga jenis berdasarkan lokasinya dalam struktur
halaman web. Yang pertama adalah \emph{fancy hit}, yaitu \emph{hit} yang berada
didalam \emph{URL}, judul halaman, atau \emph{meta tag}. Yang kedua adalah
\emph{anchor hit}, yaitu \emph{hit} yang khusus berada di dalam
\emph{anchor text} yang merupakan URL yang merujuk kepada dokumen atau halaman
lain. Yang ketiga adalah \emph{plain hit}, yaitu \emph{hit} yang berada diluar
lingkup dari \emph{fancy hit}.

Alokasi memori yang dibutuhkan untuk menyimpan sebuah \emph{hit} adalah 8 byte
atau 16 bit. Penggunaan bit tersebut berbeda tergantung dari jenis \emph{hit},
tetapi memiliki cara yang hampir sama dalam menggunakannya. 

Bit pertama digunakan sebagai penanda untuk penggunaan huruf kapital.
Selanjutnya, bit kedua hingga keempat digunakan sebagai penanda ukuran font yang
digunakan. Ukuran font ini bersifat relatif terhadap seluruh kata dalam dokumen.
Selain itu, nilai bit \emph{111} tidak digunakan sebagai ukuran penanda ukuran
font, namun digunakan sebagai penanda untuk \emph{fancy hit}. Bit kelima hingga
terakhir digunakan sebagai penanda posisi kata dalam dokumen. Karena
keterbatasan bit, posisi yang memiliki nilai lebih dari 4095 akan dianggap
bernilai 4096.

\begin{figure}[H]
  \centering{}
	\includegraphics[width=0.85\textwidth]{gambar/plain\-bit}
  \caption{Penggunaan alokasi memori dari \emph{plain hit}}
\end{figure}

Pada penggunaan \emph{fancy hit}, alokasi bit untuk penanda posisi dibagi
menjadi dua. Bit ke-9 hingga ke-12 digunakan sebagai penanda jenis dari
\emph{fancy hit}, sementara bit ke-13 hingga terakhir tetap digunakan sebagai
penanda posisi.

\begin{figure}[H]
  \centering{}
	\includegraphics[width=0.85\textwidth]{gambar/fancy\-bit}
  \caption{Penggunaan alokasi memori dari \emph{fancy hit}}
\end{figure}

Dari penggunaan \emph{fancy hit}, penggunaan \emph{anchor hit} membagi sisa
delapan bit terakhir menjadi dua, yaitu empat bit awal sebagai penanda posisi
dan empat bit terakhir sebagai nilai \emph{hash} dari dokumen dimana kata
tersebut berada.

\emph{Hit} dirangkai dengan menggunakan struktur \emph{linked list} agar
penambahan hit lebih mudah dilakukan ketika proses indexing ulang. Urutan dari
\emph{hit} dimulai dari kemunculan pertama kata tersebut dalam dokumen tertentu.
\emph{Hit} pertama dalam dokumen kemudian disambungkan ke \textit{docID} di
depannya dengan menggunakan pointer.

\begin{figure}[H]
  \centering{}
	\includegraphics[width=0.85\textwidth]{gambar/linkedListHit}
  \caption{Rangkaian \emph{hit} dengan struktur data \emph{linked list}}
\end{figure}

Iterasi terhadap seluruh \emph{hit} dapat dihindari dengan menyimpan total
\emph{hit} untuk suatu kata dalam suatu dokumen di depan rangkaian \emph{hit}.
Hal ini akan membuat proses penambahan \emph{hit} menjadi lebih efisien dan
memudahkan kalkulasi tingkat kepentingan suatu kata. Jumlah \textit{hit}
memiliki jumlah bit yang berbeda tergantung dari lokasi daftar index. Untuk 
\textit{inverted index} digunakan 5 bit, dan untuk \textit{forward index}
digunakan 8 bit.

\begin{figure}[H]
  \centering{}
	\includegraphics[width=0.85\textwidth]{gambar/hubunganStrukturPencarian}
  \caption{Penempatan jumlah \emph{hit} dalam struktur index}
\end{figure}

\subsection{Skema Penyimpanan Index}

Proses indexing akan menghasilkan \emph{forward index}, yaitu index yang telah
terurut berdasarkan \textit{docID}. Untuk membuat \textit{inverted index}, data
pada \textit{forward index} akan diurutkan berdasarkan \textit{wordID}.

Kemudian, dibuat dua salinan dari daftar index. Salinan pertama berisi daftar
\emph{hit} yang termasuk dalam kategori \emph{fancy hit} atau \emph{anchor hit}.
Salinan kedua adalah daftar \emph{hit} sisanya. Hal ini bertujuan untuk
mengurangi kemungkinan akses dari index, karena pada sebagian besar kasus query
dapat dipenuhi dengan hanya mengakses data pada \emph{fancy hit} dan
\emph{anchor hit}.

Pada contoh kasus mesin pencari, jumlah data di seluruh internet yang terus
bertambah membuat potensi penggunaan media penyimpanan menjadi hampir tidak
terbatas. Hal tersebut dapat menyebabkan tidak cukupnya kapasitas memori untuk
menampung seluruh index. Untuk mengatasinya, skema penyimpanan index perlu di
partisi menjadi beberapa bagian.

Selain itu, pada tiap salinan index, index juga akan di sortir berdasarkan
\emph{id} dari dokumen. Tujuannya adalah untuk memudahkan proses \emph{merging}
ketika dihadapkan pada situasi dimana query memiliki banyak kata.

% \subsection{Pemeringkatan Query}
%
% Untuk memenuhi kebutuhan tambahan seperti pemeringkatan query, struktur
% pencarian perlu menampung data tambahan seperti frekuensi kemunculan kata dan
% parameter kompresi. Apabila data tersebut tidak dapat dimuat seluruhnya pada
% memori, maka struktur pencarian dapat dipecah dan disimpan sebagian pada ruang
% penyimpanan. Sebagai contoh, kata yang bersifat umum dapat disimpan pada memori
% untuk mengurangi kemungkinan akses ke ruang penyimpanan.

\section{Proses Pembuatan Index}

Sebagai contoh, pada tabel \ref{tab:lirik} digunakan sampel teks berupa lirik
lagu anak-anak. Pada kasus ini, diasumsikan bahwa setiap baris merupakan isi
dari dokumen terpisah.

\begin{center}
  \begin{longtable}{ P{0.15\textwidth{}} P{0.75\textwidth{}}}
  \caption{Contoh teks dari lirik lagu anak-anak} \label{tab:lirik} \\

  % First page header
  \multicolumn{1}{c}{\textbf{Dokumen}} & \multicolumn{1}{c}{\textbf{Teks}} \\ \hline 
  \endfirsthead

  % Next page header
  \multicolumn{2}{c}%
    {{\textbf{\tablename \space{} \thetable{}:} Contoh teks dari lirik lagu anak-anak}} \\
  \multicolumn{1}{c}{\textbf{Dokumen}} & \multicolumn{1}{c}{\textbf{Teks}} \\ \hline 
  \endhead

  % Next page indication footer
  \hline \multicolumn{2}{r}{{Dilanjutkan pada halaman berikutnya}} \\ \hline
  \endfoot

  % Last footer without next page indication
  \hline \hline
  \endlastfoot

    1 & Pease porridge hot, pease porridge cold, \\
    2 & Pease porridge in the pot, \\
    3 & Nine days old. \\
    4 & Some like it hot, some like it cold, \\
    5 & Some like it in the pot, \\
    6 & Nine days old. \\
  \end{longtable}
\end{center}

Seluruh kata dalam seluruh koleksi dokumen kemudian akan diambil jumlah
kemunculannya beserta nomor dokumen tempat kata tersebut ditemukan, dan
dikumpulkan berdasarkan keunikan kata-nya.

\begin{center}
  \begin{longtable}{ P{0.1\textwidth{}} P{0.1\textwidth{}} P{0.25\textwidth{}}}
  \caption{\textit{Inverted file index} dari lirik lagu pada tabel
  \ref{tab:lirik}} \label{tab:file_index} \\

  % First page header
  \multicolumn{1}{c}{\textbf{Nomor}} & \multicolumn{1}{c}{\textbf{Kata}} & \multicolumn{1}{c}{\textbf{Index}} \\ \hline 
  \endfirsthead

  % Next page header
  \multicolumn{3}{c}%
    {{\textbf{\tablename\ \thetable{}:} \textit{Inverted file index} dari lirik lagu pada tabel
    \ref{tab:lirik}}} \\
  \multicolumn{1}{c}{\textbf{Nomor}} & \multicolumn{1}{c}{\textbf{Kata}} & \multicolumn{1}{c}{\textbf{Index}} \\ \hline 
  \endhead

  % Next page indication footer
  \hline \multicolumn{3}{r}{{Dilanjutkan pada halaman berikutnya}} \\ \hline
  \endfoot

  % Last footer without next page indication
  \hline \hline
  \endlastfoot

    1 & cold & $\langle{}2; 1, 4 \rangle{}$ \\
    2 & days & $\langle{}2; 3, 6 \rangle{}$ \\
    3 & hot & $\langle{}2; 1, 4 \rangle{}$ \\
    4 & in & $\langle{}2; 2, 5 \rangle{}$ \\
    5 & it & $\langle{}2; 4, 5 \rangle{}$ \\
    6 & like & $\langle{}2; 4, 5 \rangle{}$ \\
    7 & nine & $\langle{}2; 3, 6 \rangle{}$ \\
    8 & old & $\langle{}2; 3, 6 \rangle{}$ \\
    9 & pease & $\langle{}2; 1, 2 \rangle{}$ \\
    10 & porridge & $\langle{}2; 1, 2 \rangle{}$ \\
    11 & pot & $\langle{}2; 2, 5 \rangle{}$ \\
    12 & some & $\langle{}2; 4, 5 \rangle{}$ \\
    13 & the & $\langle{}2; 2, 5 \rangle{}$ \\
  \end{longtable}
\end{center}

Hasil index pada tabel \ref{tab:file_index} merupakan bentuk
\textit{inverted file index}, yang memetakan kata ke dokumen tempat kemunculan
kata tersebut. Untuk membuat implementasi ulang modul index milik Google, maka
diperlukan index dengan granularitas yang lebih halus. Modul index milik Google
menggunakan \textit{word-level index}, dimana index juga menyimpan posisi kata
dalam dokumen. \textit{Word-level index} dapat mengatasi situasi dimana jumlah
kemunculan suatu kata dalam dokumen lebih dari satu kali.

Dengan mengabaikan detail selain posisi kata dalam dokumen, index pada tabel
\ref{tab:file_index} dapat dikembangkan menjadi seperti berikut

\begin{center}
  \begin{longtable}{ P{0.1\textwidth{}} P{0.1\textwidth{}} P{0.2\textwidth{}}}
    \caption{\textit{Word-level index} dari lirik lagu pada tabel
    \ref{tab:lirik}}
    \label{tab:word_index} \\
    \textbf{Nomor} & \textbf{Kata} & \textbf{Index} \\
    \hline{}
    1 & cold & $\langle{}2; (1;6), (4;8) \rangle{}$ \\
    2 & days & $\langle{}2; (3;2), (6;2) \rangle{}$ \\
    3 & hot & $\langle{}2; (1;3), (4;4) \rangle{}$ \\
    4 & in & $\langle{}2; (2;3), (5;4) \rangle{}$ \\
    5 & it & $\langle{}2; (4;3,7), (5;3) \rangle{}$ \\
    6 & like & $\langle{}2; (4;2,6), (5;2) \rangle{}$ \\
    7 & nine & $\langle{}2; (3;1), (6;1) \rangle{}$ \\
    8 & old & $\langle{}2; (3;3), (6;3) \rangle{}$ \\
    9 & pease & $\langle{}2; (1;1,4), (2;1) \rangle{}$ \\
    10 & porridge & $\langle{}2; (1;2,5), (2;2) \rangle{}$ \\
    11 & pot & $\langle{}2; (2;5), (5;6) \rangle{}$ \\
    12 & some & $\langle{}2; (4;1,5), (5;1) \rangle{}$ \\
    13 & the & $\langle{}2; (2;4), (5;5) \rangle{}$ \\
  \end{longtable}
\end{center}

\subsection{Pengolahan Query}

Untuk query yang hanya mengandung satu kata, hasil bisa langsung didapatkan
dengan melakukan pencarian kosakata yang memenuhi query pada daftar kosakata.
Setelah ditemukan dokumen yang membuat kata yang memenuhi query tersebut diambil
dengan petunjuk dari data yang disematkan pada daftar kosakata.

Query yang melibatkan banyak kata akan memerlukan proses penggabungan hasil dari
kosakata yang sesuai. Sebagai contoh mudahnya, query yang melibatkan banyak kata
dapat digabungkan dengan menggunakan operator logika. Operator \textit{OR} akan
memberikan gabungan dari hasil. Sementara operator \textit{AND} akan memberikan
irisan dari hasil. Jika diberikan query sebagai berikut

\[
  Q_1 \;\textit{AND} \;Q_2 \;\textit{AND} \;\cdots{} \;\textit{AND} \;Q_N
\]

maka dilakukan pencarian untuk setiap dokumen yang mengandung kata yang memenuhi
seluruh rangkaian $Q_1$ hingga $Q_n$. Dari seluruh dokumen yang sesuai, maka
dicari irisan dari hasil tersebut (dokumen yang memiliki kedua query tersebut).

Sebagai contoh, dari index pada tabel \ref{tab:file_index}, apabila diberikan
query

\[
  hot\; \textit{AND} \;like
\]

maka akan didapatkan lokasi dari index dengan kosakata \textit{hot} dan
\textit{like}, yaitu $\langle{}1, 4 \rangle{}$ dan $\langle{}4, 5 \rangle{}$.
Dari data lokasi tersebut, mereka kemudian digabungkan (diambil irisannya),
sehingga didapatkan dokumen 4 yang memenuhi kedua query tersebut.

Untuk query yang mengandung banyak kata yang tidak digabungkan dengan operator
boolean, penggunaan \textit{inverted file} seperti contoh diatas akan
menimbulkan banyaknya \textit{false match} yang mengurangi kualitas hasil dari
query. \textit{False match} sendiri dapat diatas dengan melakukan pemindaian
untuk memastikan hasil dari query. Tetapi pemindaian akan mengurangi performa
keseluruhan dari proses pengambilan data.

Penggunaan \textit{word-level index} dapat menangani hal tersebut dengan
menggunakan informasi yang lebih lengkap yang tersemat pada tiap \textit{hit}.
Karena terdapat informasi terkait posisi kata dalam dokumen, jarak antara kata
dapat dibandingkan sesuai dengan query.

% \subsection{Query Multi Kata}
%
% Apabila query memiliki lebih dari satu kata, untuk memperoleh hasil yang lebih
% akurat, struktur pencarian membutuhkan data tambahan. Data tersebut adalah
% posisi kata dalam suatu dokumen. Dengan menyimpan posisi kata, dapat dilakukan
% perbandingan posisi antar kata yang terdapat pada query.

\section{\textit{Jaccard Distance}}

\textit{Jaccard index} adalah suatu metode statistik yang digunakan untuk
mengukur kemiripan pada suatu kumpulan sampel. \textit{Jaccard index} mengukur
kemiripan berdasarkan suatu kumpulan data yang terbatas, dan dapat dijabarkan
sebagai nilai dari perpotongan dibagi dengan \textit{union} dari data.

\begin{equation}
  J(A,B)={{|A\cap B|} \over {|A\cup B|}} = {{|A\cap B|} \over {|A| + |B| + |A\cap B|}}
\end{equation}

\textit{Jaccard distance ($d_{J}$)} adalah salah satu penerapan dari metode
statistik dengan \textit{Jaccard index}. \textit{Jaccard distance} digunakan
untuk mengukur nilai ketidakmiripan diantara dua sampel data. Nilai $d_{J}$
didapatkan dengan cara mengurangi skor $1$ dengan nilai yang didapatkan pada
perhitungan \textit{Jaccard index}.

\begin{equation}
  d_{J}(A,B)= 1 - J(A,B) = {{|A\cup B|} - {A\cap B|} \over {|A\cup B|}}
\end{equation}

\section{Modul $typing$ pada Python}

Untuk penjabaran struktur data pada modul \textit{indexing}, penulis menggunakan
bahasa pemrograman \textit{Python}. \textit{Python} dipilih karena seluruh kode
program yang ada pada arsitektur \textit{Telusuri} menggunakan bahasa yang sama.
Selain itu penulis menggunakan modul $typing$ yang mulai diperkenalkan pada
\textit{Python} versi $3.5$ untuk memberikan keterangan yang lebih jelas terkait
tipe data apa yang digunakan.

\section{Perbandingan Metode \emph{Indexing}}

\subsection{\emph{Signature File Index}}

\emph{Signature file} adalah suatu metode probabilistik yang digunakan sebagai
alternatif untuk mengindeks teks. Pada implementasi \emph{signature file},
setiap dokumen memiliki suatu \emph{descriptor}, yaitu rangkaian \emph{bit} yang
merepresentasikan konten dari dokumen. \emph{Descriptor} dibuat dengan cara
menggunakan beberapa \emph{hash function} untuk mendapatkan suatu rangkaian bit
unik yang merepresentasikan suatu kata dalam dokumen. 

Karena sifatnya yang probabilistik, \emph{descriptor} tidak dapat bersifat unik
sepenuhnya. Terdapat kemungkinan konflik antara \emph{descriptor} dari tiap
kata. Oleh karena itu, setiap hasil \emph{hashing} dari suatu kata yang sesuai
dengan \emph{descriptor} tertentu perlu diartikan sebagai suatu kemungkinan,
bukan suatu kepastian.

Untuk memastikannya, perlu dilakukan \emph{scanning} pada dokumen yang diduga
mengandung kata tersebut. Kemungkinan terjadinya konflik dapat dikurangi dengan
menggunakan lebih banyak bit, tetapi proses \emph{scanning} tetap perlu
dilakukan untuk memastikan ada atau tidaknya suatu kata.

% Signature vs Inverted
Dibandingkan dengan inverted file, signature file memiliki dua kekurangan yang
memiliki dampak langsung untuk kebutuhan mesin pencari. Yang pertama adalah
performa yang lebih lambat dalam hal pengambilan data dari database. Hal ini
dikarenakan perlunya melakukan \emph{scanning} untuk memastikan suatu kata ada
di dalam suatu dokumen. Yang kedua adalah perlunya proses yang rumit untuk
mengolah query yang mengandung disjungsi atau negasi.

Signature file memiliki satu keunggulan dibandingkan inverted file, yaitu tidak
membutuhkan penyimpanan daftar kosakata pada memori. Oleh karena itu, di masa
lalu signature file lebih umum digunakan karena alasa keterbatasan ruang
penyimpanan. Tetapi seiring berjalannya waktu, inverted file mengadopsi teknik
kompresi yang membuat keunggulan signature file ini menjadi tidak berlaku lagi.

\subsection{\emph{Bitmap Index}}

\emph{Bitmap} adalah. Bitmap menggunakan \emph{bitvector} sebagai representasi
index. Setiap kata memiliki \emph{bitvector} yang menunjukkan keberadaan kata
tersebut dalam berbagai dokumen. Panjang dari \emph{bitvector} tergantung dari
jumlah dokumen secara kesuluruhan. Konsep bitmap sangat mudah dan cepat untuk
digunakan. Tetapi, bentuk struktur \emph{bitvector} membuat kebutuhan ruang
penyimpanan menjadi sangat besar.

Karena kekurangannya dalam hal ukuran struktur data, bitmap tidak dapat
digunakan pada berbagai kebutuhan dunia nyata, terutama mesin pencari.

\subsection{\emph{WordNet}}

\emph{WordNet} adalah suatu \emph{synonym ring} yang mampu mengelompokkan kata
berdasarkan makna yang ekuivalen secara semantik (sama secara makna).
\emph{WordNet} dapat digunakan pada proses indexing dengan harapan hasil dari
pengelompokan kata yang ditemukan dalam dokumen dapat menjadi lebih baik. Hal
ini dapat memberikan peningkatan performa dalam proses pengambilan data.

Implementasi index menggunakan \emph{WordNet} dapat digunakan sebagai pembanding
untuk metode \emph{inverted file} yang telah dijelaskan sebelumnya.

% %!TEX root = ./template-skripsi.tex
%-------------------------------------------------------------------------------
%                     BAB III
%               			PEMBAHASAN
%-------------------------------------------------------------------------------

% TODO
% Update keterangan struktur data yang dipake
% - Tambahan bit capital
% - Rumus2

\chapter{METODOLOGI PENELITIAN}

\section{Tahapan Penelitian}

Gambar \textit{flowchart} berikut mengilustrasikan proses pembuatan \textit{
inverted index} dari dataset yang ada pada database hingga ke proses pencarian
oleh pengguna.

\begin{figure}[H]
  \centering{}
	\includegraphics[width=0.6\textwidth]{gambar/flowchart_awal}
  \caption{Flowchart tahapan penelitian modul \textit{indexing}}
\end{figure}

Data utama yang diperlukan oleh modul \textit{indexing} dari database adalah
teks yang berisi bagian informasi inti dari halaman tersebut. Umumnya, sebuah
halaman web memiliki banyak teks di luar informasi inti yang tidak diperlukan.
Oleh karena itu, ditetapkan bahwa modul \textit{indexing} hanya akan mengambil
bagian paragraf dari halaman web. Untuk mengidentifikasi sebuah paragraf, cukup
dengan mencari \textit{tag} HTML yang sesuai (\textit{<p>}).

Walaupun sudah menspesifikasikan \textit{tag} tertentu yang akan digunakan, pada
praktiknya sering ditemukan penggunaan \textit{tag} paragraf di luar informasi
inti halaman web. Selain itu, seringkali ditemukan paragraf yang memiliki
beberapa simbol yang tidak diperlukan seperti karakter \textit{newline}. Untuk
memastikan bahwa modul \textit{indexing} hanya akan membaca paragraf inti saja,
diperlukan \textit{filter} khusus untuk menghindari masuknya informasi yang
tidak diperlukan. Parameter dari penentuan informasi ini cukup beragam, dan akan
diatur sedemikian rupa pada masa pengujian.

Pada kode program mesin pencari yang telah ada saat ini, paragraf yang disimpan
masih memiliki banyak informasi yang tidak diperlukan. Untuk memudahkan
implementasi modul \textit{indexing}, maka proses \textit{filtering} akan
dilakukan pada kode mesin pencari di bagian ekstraksi informasi dari halaman
web. Akan dibuat sebuah table baru yang berisi paragraf inti saja.

\section{Konstruksi Index}

Sebelum proses \textit{indexing} dilakukan, diperlukan sebuah
\textit{hash table} pada memori. \textit{Hash table} digunakan dengan tujuan
untuk menyimpan kata secara unik dalam dokumen. Jika nantinya ditemukan lebih
dari satu kali kemunculan kata dalam dokumen yang sama, \textit{hash table}
dapat melakukan penambahan kepada \textit{value} dari kata tersebut, dan
membentuk suatu rangkaian seperti \textit{linked list}.

% baca webpage dan ekstrak metadata
Proses \textit{indexing} dimulai dengan mengambil seluruh daftar paragraf yang
tersimpan pada database yang diurutkan berdasarkan \textit{docID}. Kemudian,
setiap paragraf akan dipecah berdasarkan karakter spasi menjadi sebuah array
berisi kata. Setiap kata lalu disimpan dalam sebuah hit bersama dengan informasi
tambahan yang ada pada paragraf.

\begin{algorithm}[H]
  \caption{Algoritma pembuatan index (\cite{brin1998google})}\label{alg:index}
  \begin{algorithmic}
    \State $hitData \gets \Call{HitData}{false}$ \Comment{Inisiasi \textit{hash table}}
    \State $docMap \gets [..]$

    \item[] % line skip

    \State $docs[..] \gets \Call{GetDocs}$
    \For{$doc \in \mathcal{} docs[..]$}
      \State $paragraphs[..] \gets \Call{GetParagraphs}{$doc$}$
      \For{$paragraph \in \mathcal{} paragraphs[..]$}
        \State $words[..] \gets \Call{SplitWords}{$$paragraph$$}$

        \item[] % line skip

        \For{$word \in \mathcal{} words[..]$}
          \State $\Call{StoreHit}{$$hitData, word$$}$
        \EndFor

        \item[] % line skip

          \State $\Call{docMap}{$doc$} \gets hitData$
      \EndFor
    \EndFor
  \end{algorithmic}
\end{algorithm}

Dari proses ekstraksi kosakata beserta data tambahannya, akan didapatkan
\textit{forward index} berupa daftar \textit{hit} dari dokumen. Daftar
\textit{hit} kemudian didistribusikan ke tempat penyimpanan.

% sorting forward index jadi inverted
Untuk melakukan konversi \textit{forward index} menjadi \textit{inverted index},
dilakukan proses \textit{sorting} pada \textit{forward index}. \textit{Sorting}
dilakukan berdasarkan \textit{id} dari kosakata, kemudian dilanjutkan dengan
\textit{id} dari dokumen, dan diakhiri dengan posisi kata pada dokumen. Hasil
dari proses \textit{sorting} disimpan kembali ke tempat penyimpanan dengan
kondisi sudah menjadi \textit{inverted index}.

\begin{algorithm}[H]
  \caption{Operasi \textit{sorting} pada \textit{hitlist} (\cite{brin1998google})}\label{alg:insert}
  \begin{algorithmic}
    \Function{SortHitlist}{$H$}
      \State $sorted \gets false$
      \State $i \gets 0$

      \item[] % line skip

        \Do
          \If{$i = \Call{Len}{H} - 1$}
            \If{$sorted = true$}
              \State $break$
            \Else
              \State $i \gets 0$
            \EndIf
          \EndIf

          \item[] % line skip

          \State $tempHit \gets H[i]$

          \item[] % line skip

          \If{$H[i].text > H[i+1].text$ OR
          \newline \hspace*{3.8em}$H[i].docID > H[i+1].docID$ OR
          \newline \hspace*{3.8em}$H[i].oset > H[i+1].oset$}
            \State $H[i] \gets H[i+1]$
            \State $H[i+1] \gets tempHit$
          \EndIf

          \item[] % line skip

          \State $i \gets i+1$
        \doWhile{$sorted = false$}

        \item[] % line skip

      \State \Return
    \EndFunction
  \end{algorithmic}
\end{algorithm}

\section{Daftar kata umum}

Untuk meningkatkan kualitas hasil pencarian, maka skema pencarian perlu
mengabaikan kata-kata yang bersifat umum. Dari hal tersebut, perlu di dapatkan
daftar kata umum dari daftar kosakata yang telah dibuat oleh modul
\textit{indexing}. Metode yang penulis gunakan untuk mendapatkan kata umum
adalah dengan melakukan \textit{sorting} terhadap seluruh daftar kosakata
berdasarkan tingkat kemunculannya secara global. Kemudian, dengan menggunakan
batasan tertentu, akan diambil sebagian persen dari kata terbanyak yang ada.

Sebagai langkah awal, penulis berencana untuk mengambil sebanyak 1\% kata
teratas sebagai kata umum. Nilai tersebut kemudian akan diatur ulang berdasarkan
hasil \textit{user testing}.

\section{Metode Pemeringkatan}

Dari hasil rangkaian \textit{hit} yang didapatkan, perlu dilakukan beberapa
operasi untuk mendapatkan urutan hasil dokumen yang paling sesuai. Hasil
peringkat yang didapatkan dilakukan berdasarkan dua faktor berikut

\begin{itemize}
  \item{\textit{Word distance ($\gamma$)}, yaitu jarak kata yang memenuhi query}
  \item{\textit{Word similarity ($\beta$)}, yaitu kemunculan hasil yang memenuhi beberapa
    bagian dari query}
\end{itemize}

\subsection{\textit{Word distance ($\gamma$)}}

Perhitungan \textit{word distance} digunakan untuk melakukan validasi terhadap
urutan kata yang muncul pada dokumen. Urutan dari kata berpengaruh terhadap
maksud dari query. Selain itu, jarak yang terlalu jauh antara kata yang
ditemukan dapat mengurangi relevansi informasi.

Untuk membandingkan jarak kata, seluruh kata pada query perlu diberikan urutan
yang sesuai dengan menggunakan angka terlebih dahulu. Setelah \textit{hitlist}
didapatkan, posisi dari kata yang ditemukan akan dikurangi dengan posisi asli
pada query. Nilai dari dokumen bisa didapatkan dengan rumus

% TODO Revisi rumusnya dikit
\begin{equation}
  \gamma = \left(\sum_{n = 1}^{N} |L_n - \acute{L_n}|\right) - (L_1 - \acute{L_1}) \times N
  \label{eq:word_distance}
\end{equation}

di mana

\begin{conditions}
  N & Jumlah kata pada query \\
  L_n & Posisi awal kata ke-$n$ pada query \\
  \acute{L_n} & Posisi kata ke-$n$ pada query di dokumen
\end{conditions}

Ketika urutan kata pada dokumen meiliki nilai $\gamma = 0$, maka dokumen 
memenuhi kondisi \textit{exact match}. Pada kondisi tersebut, dokumen akan 
mendapatkan peringkat berdasarkan jumlah kemunculan \textit{exact match}. Sementara 
jika nilai $\gamma \not = 0$, maka dokumen akan mendapatkan peringkat 
berdasarkan rumus berikut 

\begin{equation}
  D = \frac{n}{N} + (\frac{n}{N} \times \frac{1}{G} \times K)
  \label{eq:word_distance_partial}
\end{equation}

di mana

\begin{conditions}
  n & Jumlah \textit{match} pada dokumen \\
  N & Jumlah kata pada query \\
  G & Faktor kemunculan terhadap nilai $\gamma$ \\
  K & Jumlah kemunculan \textit{partial match}
\end{conditions}

Apabila dalam suatu dokumen, terdapat berbagai kondisi \textit{match} dengan 
nilai $\gamma$ yang berbeda-beda, maka akan dipilih kondisi \textit{match} 
dengan nilai $\gamma$ paling rendah untuk merepresentasikan nilai dokumen.

\subsection{\textit{Word similarity ($\beta$)}}

\textit{Word similarity} menilai suatu kata berdasarkan kemiripannya dari kata
pada query. Untuk membandingkan apakah suatu kata memiliki makna atau maksud
yang sama dengan kata lainnya, akan digunakan \textit{Jaccard distance}.
Perbandingan akan dilakukan berdasarkan hasil kombinasi pada seluruh karakter
yang ada pada kosakata untuk membuat pasangan dua huruf.

Karena \textit{Jaccard distance} mengukur nilai ketidakmiripan dari dua model,
maka tinggi nilai $\beta$ maka nilai dokumen akan makin rendah. Dari detail
tersebut, didapatkan rumus

\begin{equation}
  \beta = 1 - \frac{m}{M}
  \label{eq:word_similarity}
\end{equation}

di mana

\begin{conditions}
  m & Jumlah pasangan dua huruf berurutan yang ada pada kata \\
  M & Jumlah kemungkinan seluruh pasangan dua huruf berurutan yang mungkin
      dibentuk \\
\end{conditions}

% TODO Ada plan untuk menggunakan cache (ini ngide aje)
Karena proses perhitungannya yang cukup kompleks, metode pemeringkatan ini hanya
digunakan ketika nilai dari \textit{word distance} bernilai kurang dari satu.

\section{Modifikasi Arsitektur \textit{Telusuri}}

\begin{figure}[H]
  \centering{}
	\includegraphics[width=0.8\textwidth]{gambar/arsitektur_baru}
  \caption{Revisi arsitektur \textit{Telusuri}}
\end{figure}

Dari arsitektur \textit{Telusuri} yang sudah ada saat ini, terdapat beberapa 
perubahan yang perlu dilakukan untuk mengintegrasikan modul \textit{indexing}
ini.

\begin{itemize}
  \item{\textit{Telusuri} dibagi menjadi dua mode \textit{runtime}, yaitu 
    \textit{crawling} dan \textit{indexing}.}
  \item{Kontrol terhadap berjalannya \textit{Telusuri} akan dikendalikan oleh 
    aplikasi \textit{command-line}, yaitu \textit{telusurictl}. \textit{telusurictl} 
    akan berkomunikasi dengan \textit{Telusuri} melalui sebuah metode komunikasi 
    antar proses. Untuk metode komunikasi akan digunakan, ditentukan untuk 
    menggunakan \textit{UNIX domain socket} dengan alasan didukung secara 
    langsung oleh sistem operasi yang digunakan (Linux) dan bersifat
    \textit{language-agnostic} sehingga mudah untuk dikembangkan kedepannya.}
  \item{Terdapat modul database primitif yang berperan sebagai \textit{persistent 
    storage} untuk \textit{hitlists} yang telah dibuat oleh modul \textit{indexing}.
    Modul database ini berjalan sebagai proses terpisah, tetapi tetap dalam satu 
    mesin yang sama. Modul akan memberikan \textit{hitlists} sesuai dengan 
    rentang yang diminta oleh modul \textit{indexing}. Untuk komunikasi antar 
    modul, akan digunakan \textit{UNIX domain socket}.}
\end{itemize} 

\section{Skema Pencarian}

Untuk mendapatkan informasi pencarian berdasarkan input teks dari pengguna,
diperlukan beberapa informasi tertentu yang bisa didapatkan dengan cara mengolah 
informasi yang ada pada teks tersebut. Objek \textit{UserQuery} digunakan untuk 
mengakomodasi kebutuhan itu.

\begin{figure}[H]
  \centering{}
	\includegraphics[width=0.8\textwidth]{gambar/struktur_userquery}
  \caption{Struktur objek \textit{UserQuery}}
\end{figure}

% Parsing query
Berdasarkan input query yang diberikan oleh pengguna, query dipecah menjadi
potongan kata dan ditetapkan beberapa informasi tambahan bedasarkan 
karakteristik dari masing-masing kata. Informasi tersebut adalah posisi kata 
dalam teks input, apakah kata tersebut termasuk kata umum dan apakah kata 
tersebut termasuk kapital. Ketiga informasi ini akan disimpan sebagai
\textit{WordInfo} yang ada pada objek \textit{UserQuery}.

\begin{figure}[H]
  \centering{}
	\includegraphics[width=0.85\textwidth]{gambar/flowchart_pencarian}
  \caption{Flowchart pengolahan masukan query}
\end{figure}

Untuk setiap kosakata pada daftar kosakata yang sesuai dengan kosakata pada
query, akan di dapatkan rangkaian data berupa \textit{id} dari dokumen tempat
kosakata tersebut, total jumlah \textit{hit} pada dokumen serta seluruh daftar
\textit{hit} yang berhubungan dengan kosakata pada dokumen.

\begin{figure}[H]
  \centering{}
	\includegraphics[width=1\textwidth]{gambar/struktur_data_gambar}
  \caption{Ilustrasi hubungan struktur data}
\end{figure}

\begin{figure}[H]
  \centering{}
	\includegraphics[width=0.9\textwidth]{gambar/struktur_data_python}
  \caption{Penulisan struktur data pada \textit{Python}}
\end{figure}

Untuk setiap daftar \textit{hit} yang didapatkan, dilakukan \textit{merging}
sehingga tersisa dokumen yang memenuhi query.

\begin{algorithm}[H]
  \label{alg:merge}
  \caption{Operasi \textit{merging} pada seluruh \textit{hitlists} (\cite{brin1998google})}
  \begin{algorithmic}
    \Function{MergeHitlist}{$p1$, $p2$}
    \State $M \gets [..]$
    \While{$p1 \not = nil$ AND $p2 \not = nil$}
      % TODO Handle multiline
      \If{$p1.docID = p2.docID$} \Comment{Cek kesamaan \textit{docID}}
        \If{$-(p1.oset - p2.oset)$ = 1} \Comment{Cek selisih offset}
          \State \Call{Append}{$M$, $p1$}
        \EndIf
      \EndIf
      \State $p1 \gets \Call{next}{$p1$}$
      \State $p2 \gets \Call{next}{$p2$}$
    \EndWhile{}
    \State \Return $M$
    \EndFunction

    \item[] % line skip

    \State $H \gets [..]$
    \State $W \gets \Call{ParseQuery}{query}$ \Comment{Pecah query menjadi
    daftar kata}
    \For{$w \in \mathcal{} W$}
      \State $h \gets \Call{GetHitlist}{w}$
      \If{$h = nil$}
        \State continue
      \EndIf
      \State \Call{Append}{H, h}
    \EndFor

    \item[] % line skip

    \State $i \gets 0$
    \While{$i \not = \Call{Len}{H}$}
      \State $H[0] = \Call{MergeHitlist}{$H[0]$, $H[i]$}$
      \State $i \gets i + 1$
    \EndWhile

    \item[] % line skip

    \State $result \gets \Call{GetResult}{H[0]}$
  \end{algorithmic}
\end{algorithm}

\section{Alat dan Bahan Penelitian}

Pada penelitian ini, terdapat beberapa alat yang digunakan sebagai penunjang
dalam pembuatan modul \textit{indexing} dengan rincian sebagai berikut:

\begin{itemize}
  \item{Laptop dengan konfigurasi Intel Core i7-8650U, 32GB RAM)}
  \item{Sistem operasi \textit{Linux}}
  \item{\textit{Neovim} sebagai \textit{code editor}}
  \item{Database \textit{MySQL} versin $8.0.31$}
  \item{\textit{Podman} untuk menjalankan database \textit{MySQL}}
  \item{\textit{Python 3.8}}
\end{itemize}

\section{Tahapan Pengembangan}

\subsection{Meningkatkan kemampuan modul \textit{indexing} saat ini}

Pencarian akan dilakukan secara sekuensial terhadap daftar index yang telah
dihasilkan oleh modul \textit{indexing}. Sebagai pembanding, modul dari
penelitian ini akan digabungkan dengan modul \textit{indexing} yang sudah ada 
saat ini (\cite{zaidan2023gst}).

Hasil penelitian tersebut menghasilkan modul \textit{indexing} dengan bentuk
\textit{Generalized Suffix Tree (GST)}. Penggunaan modul \textit{GST} bertujuan
untuk menghindari proses iterasi daftar kosakata yang bisa memakan waktu yang
cukup lama. Modul \textit{GST} dapat langsung memberikan hasil berupa dokumen
tempat kata tersebut berada. Dengan demikian, modul index pada penelitian ini
hanya perlu menunjukkan lokasi kata pada dokumen melalui informasi yang ada
pada \textit{hit}. Selain itu, penggunaan \textit{GST} juga dapat melakukan
koreksi terhadap adanya \textit{typo} pada query yang diberikan oleh pengguna.

Untuk mendukung integrasi dengan modul \textit{GST}, diperlukan pemetaan dari
dokumen ke \textit{hitlists} yang sesuai. Oleh karena itu, akan dilakukan 
modifikasi pada \ref{alg:index} untuk menambahkan pemetaan dokumen ke
\textit{hitlists} yang terbentuk.

\begin{figure}[H]
  \centering{}
	\includegraphics[width=0.8\textwidth]{gambar/flowchart_pencarian_gst}
  \caption{Flowchart pengolahan masukan query dengan modul \textit{GST}}
\end{figure}

Integrasi dilakukan dengan modul \textit{GST} yang telah dibuat pada penelitian 
sebelumnya, dengan beberapa perbaikan dan penambahan fungsi pembantu untuk 
mempermudah integrasi.

\subsection{Rancangan Eksperimen}

\begin{enumerate}
  \item{Skenario pertama (Pencarian degnan iterasi daftar kosakata)}
    \begin{itemize}
      \item{Pencatatan waktu akan mulai mencatat}
      \item{\textit{Tester} memberikan sebuah query ke dalam mesin pencari}
      \item{Sistem memecah kata berdasarkan spasi}
      \item{Sistem melakukan iterasi pada daftar kosakata untuk mencari kata 
        yang sesuai dengan query}
      \item{Sistem melakukan pencarian pada kata dengan mencari pada daftar 
        kosakata}
      \item{Sistem menggabungkan seluruh hasil pencarian per kata}
      \item{Sistem melakukan pemeringkatan hasil}
      \item{Sistem mengembalikan daftar dokumen hasil pencarian yang diurutkan
        berdasarkan hasil pemeringkatan}
      \item{Pencatatan waktu berhenti mencatat}
      \item{\textit{Tester} memberikan penilaian terhadap relevansi dari hasil
        pencarian}
      \item{Hasil perhitungan waktu akan dibandingkan dengan skenario lain}
    \end{itemize}
  \item{Skenario kedua (Pencarian dengan integrasi modul GST)}
    \begin{itemize}
      \item{Pencatatan waktu akan mulai mencatat}
      \item{\textit{Tester} memberikan sebuah query ke dalam mesin pencari}
      \item{Sistem memecah kata berdasarkan spasi}
      \item{Sistem melakukan input kata ke modul \textit{GST} dan mendapatkan 
        output berupa hasil pencarian}
      \item{Sistem menggabungkan seluruh hasil pencarian per kata}
      \item{Sistem melakukan pemeringkatan hasil}
      \item{Sistem mengembalikan daftar dokumen hasil pencarian yang diurutkan
        berdasarkan hasil pemeringkatan}
      \item{Pencatatan waktu berhenti mencatat}
      \item{\textit{Tester} memberikan penilaian terhadap relevansi dari hasil
        pencarian}
      \item{Hasil perhitungan waktu akan dibandingkan dengan skenario lain}
    \end{itemize}
\end{enumerate}

% %!TEX root = ./template-skripsi.tex
%-------------------------------------------------------------------------------
%                            	BAB IV
%               		IMPLEMENTASI DAN PENGUJIAN
%-------------------------------------------------------------------------------

\chapter{HASIL DAN PEMBAHASAN}

\section{Implementasi}

Berikut adalah keterangan \textit{flowchart} tentang apa saja yang telah 
berhasil diimplementasikan. Warna hijau menandakan implementasi telah berhasil 
dibuat, sementara warna merah menandakan implementasi belum sempurna.

\begin{figure}[H]
  \centering{}
	\includegraphics[width=0.7\textwidth]{gambar/flowchart_jadi}
  \caption{\textit{Flowchart} tahapan penelitian yang sudah jadi}
\end{figure}

Untuk penyimpanan persisten saat ini berjalan, tetapi tidak secara
\textit{concurrent} karena keterbatasan kemampuan penulis.

\subsection{Pengolahan \textit{Dataset}}

Sebelum menjalankan modul \textit{indexer}, dataset perlu dikumpulkan terlebih 
dahulu. Proses pengumpulan dataset dilakukan dengan menggunakan modul
\textit{crawler} dari hasil penelitian sebelumnya. Titik pengumpulan dataset 
adalah situs \textit{Forum News Network} (\textit{fnn.co.id}), dengan durasi
\textit{crawling} selama 48 jam. Modul \textit{crawler} dijalankan pada hari
Jum'at, 23 Juni 2023 dan selesai pada hari Minggu, 25 Juni 2023.

Pada database, terdapat dua tabel yang akan digunakan dalam modul
\textit{indexer}. Pertama adalah tabel \textit{page\_paragraph} yang berisi 
seluruh teks dalam \textit{tag} paragraf pada halaman web. Data dari tabel ini 
akan digunakan untuk membentuk struktur index yang akan digunakan dalam proses 
pencarian data. Yang kedua adalah tabel \textit{page\_information} yang berisi
informasi tentang judul dan \textit{URL} dari halaman web. Data dari tabel ini 
akan digunakan untuk mendapatkan detail dari halaman setelah proses kalkulasi 
skor tiap dokumen.

\begin{figure}[H]
  \centering{}
	\includegraphics[width=0.7\textwidth]{gambar/implementasi_statistik_dataset}
  \caption{Statistik dataset hasil \textit{crawling}}
\end{figure}

\begin{figure}[H]
  \centering{}
	\includegraphics[width=0.85\textwidth]{gambar/implementasi_pageparagraph}
  \caption{Informasi pada tabel \textit{page\_paragraph}}
\end{figure}

\subsection{Pembentukan \textit{Inverted Index}}

Fungsi \textit{getRepoDump} mengambil dataset dari database dengan menggunakan 
objek \textit{Database} yang sudah diinisiasi sebelumnya. Kemudian didapatkan
sebuah \textit{list} berisi pasangan \textit{id} dari dokumen dan paragraf yang
ditemukan pada dokumen tersebut.

\begin{figure}[H]
  \centering{}
	\includegraphics[width=0.85\textwidth]{gambar/implementasi_getrepodump}
  \caption{Pengambilan dataset dari database}
\end{figure}

Setelah dataset didapatkan, fungsi \textit{generateIndex} akan membuat struktur 
index berdasarkan data tersebut. Untuk setiap pasangan pada dataset, paragraf 
akan dikumpulkan berdasarkan \textit{id} dari dokumen. Setelah pemetaan antara 
dokumen dan paragraf selesai dibuat, maka untuk setiap dokumen daftar paragraf 
akan diolah oleh fungsi \textit{generateHitlists}.

\begin{figure}[H]
  \centering{}
	\includegraphics[width=0.85\textwidth]{gambar/implementasi_generateindex}
  \caption{Pembuatan struktur index dari dataset}
\end{figure}

Pada fungsi \textit{generateHitlists}, setiap paragraf akan dibersihkan terlebih 
dahulu sehingga hanya tersisa karakter alfanumerik yang dipisahkan oleh spasi.
Paragraf yang sudah dibersikan kemudian akan dibagi berdasarkan spasi menjadi 
sebuah daftar kata. Kemudian, untuk setiap kata akan disimpan informasi tentang 
\textit{id} dari dokumen saat ini, posisi kata pada dokumen, dan apakah kata 
tersebut termasuk kapital (singkatan dari sebuah istilah).

\begin{figure}[H]
  \centering{}
	\includegraphics[width=0.85\textwidth]{gambar/implementasi_generatehitlists}
  \caption{Pengolahan paragraf menjadi \textit{hitlist}}
\end{figure}

Fungsi \textit{storeIndex} digunakan untuk menyimpan struktur index pada sebuah
file persisten dengan nama \textit{telusuri\_store.pkl}. File ini nantinya dapat 
langsung digunakan kembali ketika modul tidak menggunakan konfigurasi
\textit{reindex}.

\begin{figure}[H]
  \centering{}
	\includegraphics[width=0.85\textwidth]{gambar/implementasi_storeindex}
  \caption{Penyimpanan struktur index ke file persisten}
\end{figure}

\subsection{Pengolahan \textit{User Query}}

Setelah menerima input \textit{query} dari pengguna, teks akan diolah pada 
fungsi \textit{getInputPairs}. Fungsi tersebut akan memecah teks input dari 
pengguna berdasarkan spasi, menjadi sebuah daftar kata yang berurutan. Untuk 
setiap kata pada teks input, akan diolah informasi relatif terhadap teks input.
Informasi tersebut adalah posisi kata pada teks input, apakah kata tersebut 
masuk dalam kategori kata umum dan apakah kata tersebut adalah kata kapital.
Seluruh informasi disimpan dalam sebuah objek \textit{UserQuery}, yang nantinya 
akan menyimpan seluruh informasi yang telah diolah berdasarkan input pengguna.

\begin{figure}[H]
  \centering{}
	\includegraphics[width=\textwidth]{gambar/implementasi_search}
  \caption{Fungsi pencarian utama}
\end{figure}

\begin{figure}[H]
  \centering{}
	\includegraphics[width=\textwidth]{gambar/implementasi_getinputpairs}
  \caption{Pengolahan teks input dari pengguna}
\end{figure}

Dalam proses pengolahan teks input, untuk setiap kata yang bersifat umum maka 
\textit{hitlist} tidak akan diambil untuk digabungkan karena alasan efisiensi.
Selain itu, untuk kata yang bersifat kapital, apabila kata tersebut tidak ada 
pada daftar kosakata, maka akan dicoba untuk didapatkan varian non-kapitalnya 
terlebih dahulu. Seluruh kata yang sudah didapatkan \textit{hitlist}-nya, 
posisi relatifnya akan dimasukkan ke dalam variabel \textit{expectedPos} untuk 
dibandingkan ketika proses pemeringkatan. Proses ini dilakukan pada fungsi 
\textit{generateExpectedPos}.

\begin{figure}[H]
  \centering{}
	\includegraphics[width=\textwidth]{gambar/implementasi_generateexpectedpos}
  \caption{Fungsi untuk menyimpan posisi yang sesuai dengan teks input 
  berdasarkan urutan}
\end{figure}

Setelah \textit{hitlist} didapatkan pada masing-masing kata, seluruh data akan
digabungkan menjadi satu \textit{hitlist} lengkap, dan kemudian diurutkan dari 
nilai terkecil.

\subsection{Proses Penggabungan \textit{Hitlist} dan Pemeringkatan Hasil}

Pada fungsi \textit{calculateRanking}, akan dilakukan iterasi pada seluruh
\textit{hitlist} untuk menghitung nilai per dokumen. Untuk setiap iterasi,
setiap posisi akan disimpan dalam variabel \textit{curIter}, dan akan dicek 
apakah isi dari variabel tersebut telah sesuai dengan \textit{expectedPos}.
Apabila iterasi telah pindah ke dokumen lain sebelum seluruh kata terkumpul, 
maka dokumen sebelumnya akan masuk dalam golongan \textit{substring match}.
Selain jenis \textit{match}, jumlah kemunculan \textit{match} tersebut juga akan 
disimpan, dan dihitung ketika terjadi perpindahan dokumen pada iterasi.

\begin{figure}[H]
  \centering{}
	\includegraphics[width=\textwidth]{gambar/implementasi_calculaterank}
  \caption{Proses kalkulasi peringkat}
\end{figure}

Setelah proses pemeringkatan selesai, dokumen akan disaring terlebih dahulu 
berdasarkan variabel \textit{documentBlacklist} melalui fungsi
\textit{filterQuery}. Hasil yang telah disaring akan digunakan untuk mendapatkan
data halaman web dari database. Melalui fungsi \textit{getDocuments}, akan
diambil seluruh data yang mengandung nilai dari seluruh \textit{id} dokumen.
Kemudian, data akan dicetak berdasarkan peringkat dokumen tertinggi.

\begin{figure}[H]
  \centering{}
	\includegraphics[width=0.92\textwidth]{gambar/word_count}
  \caption{Grafik jumlah kata dalam dokumen}
\end{figure}

\begin{figure}[H]
  \centering{}
	\includegraphics[width=0.7\textwidth]{gambar/word_count2}
  \caption{Grafik jumlah kata dalam dokumen setelah menyingkirkan 20 entri 
  tertinggi}
\end{figure}

\begin{figure}[H]
  \centering{}
	\includegraphics[width=0.85\textwidth]{gambar/implementasi_filterquery}
  \caption{Penyaringan terhadap dokumen yang telah di-\textit{blacklist}}
\end{figure}

\begin{figure}[H]
  \centering{}
	\includegraphics[width=\textwidth]{gambar/implementasi_getdocuments}
  \caption{Pengambilan dokumen dari database dan pemetaan informasi dokumen 
  berdasarkan skor}
\end{figure}

\subsection{Integrasi dengan \textit{Generalized Suffix Tree (GST)}}

Implementasi \textit{GST} yang dipilih adalah implementasi dari hasil penelitian
sebelumnya (\cite{zaidan2023gst}). Tetapi karena kondisi kode yang tidak bisa
langsung digunakan, diperlukan beberapa perbaikan pada kode. Bagian kode yang
perlu diperbaiki adalah penggunaan iterasi yang lebih tepat dan beberapa
penambahan operasi kondisional untuk menangani kesalahan pada data.

Perubahan pertama adalah pada kondisi yang digunakan untuk iterasi ketika 
mengambil informasi dari database. Ditambahkan operasi kondisional untuk 
mengecek apakah kolom \textit{title} tidak bernilai \textit{None} dan memiliki 
panjang lebih dari $0$.

\begin{figure}[H]
  \centering{}
	\includegraphics[width=0.7\textwidth]{gambar/implementasi_gstchange2}
  \caption{Perubahan pada iterasi ketika mengambil data dari database}
\end{figure}

Perubahan kedua adalah pada kondisi yang digunakan untuk interasi ketika membuat 
struktur \textit{tree}. Ditambahkan operasi kondisional untuk memastikan bahwa 
nilai dari \textit{title} tidak kosong.

\begin{figure}[H]
  \centering{}
	\includegraphics[width=0.7\textwidth]{gambar/implementasi_gstchange3}
  \caption{Perubahan pada iterasi ketika membentuk struktur \textit{tree}}
\end{figure}

Setelah implementasi diperbaiki dan dikonfirmasi dapat berjalan melalui 
pengujian terpisah, dilakukan \textit{refactoring} pada kode untuk memudahkan 
integrasi. \textit{Class GST} dibuat, dan ditambahkan beberapa fungsi untuk 
memudahkan integrasi. Modul \textit{indexer} hanya perlu melakukan impor modul 
untuk dapat menggunakannya.

\begin{figure}[H]
  \centering{}
	\includegraphics[width=0.7\textwidth]{gambar/implementasi_gst}
  \caption{Implementasi \textit{class GST} untuk memudahkan integrasi}
\end{figure}

Selain itu, agar lebih mudah dalam proses pengujian, implementasi diberikan 
sebuah \textit{guard} kondisional agar dapat dinyalakan / dimatikan dengan 
menggunakan \textit{environment variable}.

Untuk mengakomodasi implementasi \textit{GST}, terdapat beberapa penambahan 
kode. Yang pertama adalah pemetaan antara dokumen dengan \textit{hitlists}. Untuk 
itu dibuat variabel \textit{documentPairs}. Variabel ini akan menambahkan
\textit{hit} dalam fungsi \textit{generateIndex}, tetapi akan ditambahkan per 
dokumen.

\begin{figure}[H]
  \centering{}
	\includegraphics[width=0.7\textwidth]{gambar/implementasi_gst_init}
  \caption{Penambahan inisiasi variabel untuk akomodasi penggunaan \textit{GST}}
\end{figure}

Selanjutnya, pada proses pengolahan \textit{query}, dibuat sebuah fungsi baru 
yaitu \textit{calculateRankingGST}. Karena bentuk input yang berupa daftar
\textit{hitlists} berdasarkan pemetaan dokumen, maka proses perhitungan 
peringkat agak berubah. Proses perhitungan menghindari iterasi secara menyeluruh 
dari \textit{hitlist} dengan cara mengambil irisan dari \textit{hit} yang sesuai 
dengan kata dan dokumen. Dari hasil irisan yang sesuai, barulah perhitungan 
peringkat dimulai.

\begin{figure}[H]
  \centering{}
	\includegraphics[width=0.7\textwidth]{gambar/implementasi_gst_calculateranking}
  \caption{Perubahan implementasi perhitungan peringkat untuk integrasi
  \textit{GST}}
\end{figure}

Yang terakhir adalah penambahan fungsi untuk mengambil \textit{hitlist} dari 
pemetaan dokumen. Fungsi \textit{getDocumentPairs} dimulai dengan melakukan 
input kata ke modul \textit{GST}. Untuk setiap kata, modul \textit{GST} akan 
mengembalikan daftar dokumen yang memiliki kata tersebut beserta jumlah-nya. 
Dokumen kemudian akan diurutkan berdasarkan jumlah kata, dan untuk setiap 
dokumen akan diambil \textit{hitlist} yang memenuhi.

\begin{figure}[H]
  \centering{}
	\includegraphics[width=0.7\textwidth]{gambar/implementasi_gst_getdocumentpairs}
  \caption{Fungsi untuk mengambil dokumen berdasarkan hasil dari \textit{GST}}
\end{figure}

\subsection{Modifikasi kode \textit{TF-IDF}}

Metode \textit{TF-IDF} digunakan sebagai patokan untuk memverifikasi kesesuaian 
kata pada hasil pencarian. Kode yang berdasarkan implementasi pemeringkatan 
dokumen pada penelitian sebelumnya. Kode tersebut tidak bisa dijalankan secara 
langsung karena penggunaan \textit{dataset} yang besar membuat proses skoring 
kata dalam dokumen terlalu lama.

Modifikasi yang dilakukan adalah dengan membagi operasi menjadi beberapa bagian, 
dan menyimpan hasil dari setiap operasi kedalam file persisten untuk menghindari 
melakukan proses ulang data.

Operasi pertama yang dilakukan adalah melakukan penyaringan terhadap dataset
yang ada. Agar metode \textit{TF-IDF} dapat bekerja dengan baik, maka dataset
perlu dibersihkan dari kata sambung, atau disisakan kata dasar saja. Metode yang
dipilih adalah penyaringan untuk menyisakan kata dasar saja. Untuk mengetahui 
apakah sebuah kata termasuk kata dasar, diperlukan sebuah daftar yang berisi 
kata dasar untuk digunakan sebagai patokan. Daftar kata yang digunakan diambil 
dari proyek \textit{Sastrawi} (\cite{sastrawi}).

\begin{figure}[H]
  \centering{}
	\includegraphics[width=\textwidth]{gambar/implementasi_filterkatadasar.png}
  \caption{Operasi filter pada seluruh dataset dari \textit{table\_information} 
  untuk menyisakan seluruh kemunculan kata dasar pada tiap dokumen}
\end{figure}

Pada percobaan awal, proses filter secara langsung masih memakan waktu yang 
cukup signifikan karena daftar kata dasar yang cukup besar (29932 kata). Oleh 
karena itu, dibuat pemetaan antara huruf pertama dari kata dasar ke seluruh kata 
yang memiliki huruf pertama yang sama.

Setelah terbentuk \textit{dataset} yang telah di filter, seluruh data tersebut 
dimasukkan ke dalam database. Untuk menghindari hal yang tidak diinginkan, 
dibuat salinan tabel \textit{page\_information} ke tabel baru
\textit{page\_information\_filtered} yang nantinya akan menjadi sumber data 
pengganti. Kolom \textit{content\_text} kemudian diperbarui dengan data yang 
sudah di filter.

\begin{figure}[H]
  \centering{}
	\includegraphics[width=\textwidth]{gambar/implementasi_pageinfo_filtered.png}
  \caption{Pengisian tabel \textit{page\_information\_filtered} dengan data yang 
  sudah di filter dengan daftar kata dasar}
\end{figure}

Operasi selanjutnya adalah pembuatan peringkat kata per dokumen. Pada 
implementasi sebelumnya, setiap kata yang telah mendapatkan skor akan segera di 
masukkan nilainya ke database. Karena proses ini diidentifikasi sebagai proses 
yang paling memperlambat pembuatan skoring kata, maka operasi database dipisah 
dan dilakukan setelah seluruh skor kata per dokumen terbentuk.

\begin{figure}[H]
  \centering{}
	\includegraphics[width=\textwidth]{gambar/implementasi_generate_tfidf.png}
  \caption{Proses skoring kata dari data yang sudah di filter}
\end{figure}

\begin{figure}[H]
  \centering{}
	\includegraphics[width=\textwidth]{gambar/implementasi_insert_tfidfword.png}
  \caption{Pengisian tabel \textit{tfidf\_word} dari file persisten}
\end{figure}

\begin{figure}[H]
  \centering{}
	\includegraphics[width=0.7\textwidth]{gambar/implementasi_statistik_dataset_filtered}
  \caption{Statistik dataset setelah proses \textit{filtering}}
\end{figure}

\subsection{Struktur Direktori Kode}

Struktur penulisan kode yang digunakan mengikuti struktur yang telah ada pada 
penelitian sebelumnya. Pada direktori \textit{src}, dibuat folder baru dengan 
nama \textit{indexing} sebagai penanda modul baru. Kemudian di dalamnya terdapat 
file \textit{inverted\_index.py} sebagai tempat implementasi utama.

Untuk menjalankan modul \textit{indexer}, pada direktori utama dibuat file
\textit{run\_index.py} yang memanggil objek dari modul \textit{inverted\_index}.

\section{Pengujian}

\subsection{Statistik Pengujian}

Pengujian dimulai dengan merekam waktu yang terpakai dalam proses pembuatan
struktur index berdasarkan data dari database.

\begin{table}[H]
\begin{center}
  \caption{\label{tabel:durasi_reindex} Durasi proses \textit{reindexing}}
\begin{tabular}{|c|c|c|c|} 
 \hline
  \textit{Mode} & Percobaan Ke-1 & Percobaan Ke-2 & Percobaan Ke-3 \\ 
 \hline
  Pengambilan \textit{dataset} & 7.6138s & 7.5740s & 7.5910s \\
  Inverted Index & 204.2576s & 209.5963s & 204.3641s \\ 
  Inverted Index + GST & 526.0686s & 527.3835s & 535.1130s \\
 \hline
\end{tabular}
\end{center}
\end{table}

Karena terdapat peningkatan signifikan dalam pembuatan index dengan integrasi 
\textit{GST}, maka direkam waktu pembuatan untuk struktur \textit{GST} sendiri.

\begin{table}[H]
\begin{center}
  \caption{\label{tabel:durasi_gst} Durasi pembuatan struktur \textit{GST}}
\begin{tabular}{|c|c|} 
 \hline
  Percobaan & Durasi \\ 
 \hline
  1 & 302.1005s \\ 
  2 & 308.6661s \\
  3 & 319.0878s \\
 \hline
\end{tabular}
\end{center}
\end{table}

Selanjutnya adalah merekam waktu yang digunakan untuk menyimpan struktur yang 
telah dibuat ke dalam file persisten dengan modul \textit{pickle}.

\begin{table}[H]
\begin{center}
  \caption{\label{tabel:durasi_store} Durasi proses penyimpanan struktur data 
  ke file persisten}
\begin{tabular}{|c|c|c|c|} 
 \hline
  \textit{Mode} & Percobaan Ke-1 & Percobaan Ke-2 & Percobaan Ke-3 \\ 
 \hline
  Inverted Index & 6.3289s & 6.1939s & 6.2439s \\ 
  Inverted Index + GST & 6.2494s & 6.2009s & 6.6043s \\
 \hline
\end{tabular}
\end{center}
\end{table}

Setelah itu, direkam proses muat ulang dari file persisten ke struktur data asli
pada memori sebelum digunakan untuk pencarian.

\begin{table}[H]
\begin{center}
  \caption{\label{tabel:durasi_restore} Durasi proses muat ulang struktur data 
  dari file persisten}
\begin{tabular}{|c|c|c|c|} 
 \hline
  Struktur & Percobaan Ke-1 & Percobaan Ke-2 & Percobaan Ke-3 \\ 
 \hline
  Pemetaan Kata -- \textit{Hitlists} & 2.2921s & 2.3181s & 2.2682s \\
  Pemetaan Dokumen -- \textit{Hitlists} & 1.9319s & 1.9767s & 1.9284s \\ 
  GST & 2.1220s & 2.1438s & 2.1274s \\
 \hline
\end{tabular}
\end{center}
\end{table}

Berikut adalah beberapa statistik penggunaan memori dan media penyimpanan yang
dikumpulkan selama pengujian berlangsung.

\begin{table}[H]
\begin{center}
  \caption{\label{tabel:ukuran_struktur} Ukuran struktur data yang terbentuk}
\begin{tabular}{|c|c|c|} 
 \hline
  Struktur & Ukuran di memori & Ukuran file persisten \\ 
 \hline
  Pemetaan Kata -- \textit{Hitlists} & 2035.57 MB & 264 MB \\ 
  Pemetaan Dokumen -- \textit{Hitlists} & 2026.64 MB & 263 MB \\
  GST & 93.75 MB & 7.3 MB \\
 \hline
\end{tabular}
\end{center}
\end{table}

\subsection{Pengujian Relevansi}

Berikut adalah kata kunci yang digunakan untuk uji relevansi hasil pencarian.
Kata kunci dipilih karena dapat digunakan dengan metode \textit{TF-IDF} sebagai 
patokan. Implementasi \textit{TF-IDF} sendiri sepertinya masih memiliki masalah 
di mana terdapat sekumpulan kata yang tidak menghasilkan output apa pun dengan 
menggunakan \textit{dataset} yang digunakan. Sempat dilakukan pengujian dengan 
menggunakan \textit{dataset} baru yang lebih kecil, dan berhasil. Tetapi tetap 
gagal untuk \textit{dataset} utama.

\begin{center}
\begin{longtable}{|c|c|} 
  \caption{\label{tabel:daftar_query} \textit{Query} yang digunakan untuk uji 
  relevansi}\\
 \hline
  \textit{Query} satu kata & \textit{Query} dua kata \\ 
 \hline
  bupati & ganjar pranowo \\ 
  waskita & tiket coldplay \\
  nuklir &  \\
 \hline
\end{longtable}
\end{center}

% Berikut adalah hasil pencarian dengan kata kunci \textit{bupati}.

\begin{center}
\begin{longtable}{|c|c|} 
  \caption{\label{tabel:hasil_inv_bupati} Hasil pencarian dengan kata kunci
  \textit{bupati} dengan metode \textit{inverted index}}\\

  \hline
  % First page header
  \multicolumn{1}{|c|}{\textbf{Peringkat}} & \multicolumn{1}{|c|}{\textbf{Hasil}}\\ \hline 
  \endfirsthead

  % Next page header
  \multicolumn{2}{c}%
    {{\textbf{\tablename\ \thetable{}:} Hasil pencarian dengan kata kunci
  \textit{bupati} dengan metode \textit{inverted index} }} \\
  \hline
  \multicolumn{1}{|c|}{\textbf{Peringkat}} & \multicolumn{1}{|c|}{\textbf{Hasil}}\\ \hline 
  \endhead

  % Next page indication footer
  \hline \multicolumn{2}{|r|}{{Dilanjutkan pada halaman berikutnya}} \\ \hline
  \endfoot

  % Last footer without next page indication
  \hline \hline
  \endlastfoot

 \hline
  1 & Mochamad Toha \\ 
   & \textit{http://fnn.co.id/author/Mochamad Toha} \\
 \hline
  2 & daerah \\
   & \textit{http://fnn.co.id/category/daerah?page=5} \\
 \hline
  3 & kesehatan \\
   & \textit{http://fnn.co.id/category/kesehatan?page=13} \\
 \hline
  4 & nasional \\
   & \textit{http://fnn.co.id/category/nasional?page=11} \\
 \hline
  5 & daerah \\
   & \textit{http://fnn.co.id/category/daerah?page=11} \\
 \hline
\end{longtable}
\end{center}

\begin{center}
  \begin{longtable}[c]{|p{0.2\textwidth}|p{0.8\textwidth}|} 
  \caption{\label{tabel:hasil_gst_bupati} Hasil pencarian dengan kata kunci
  \textit{bupati} dengan metode \textit{inverted index} dan integrasi dengan
  \textit{GST}} \\

  \hline
  % First page header
    \multicolumn{1}{|p{0.2\textwidth}|}{\textbf{Peringkat}} & \multicolumn{1}{|p{0.8\textwidth}|}{\textbf{Hasil}}\\ \hline 
  \endfirsthead

  % Next page header
    \multicolumn{2}{p{\textwidth}}%
    {{\textbf{\tablename\ \thetable{}:} Hasil pencarian dengan kata kunci
  \textit{bupati} dengan metode \textit{inverted index} dan integrasi dengan
    \textit{GST}}} \\
  \hline
    \multicolumn{1}{|p{0.2\textwidth}|}{\textbf{Peringkat}} & \multicolumn{1}{|p{0.8\textwidth}|}{\textbf{Hasil}}\\ \hline 
  \endhead

  % Next page indication footer
  \hline \multicolumn{2}{|r|}{{Dilanjutkan pada halaman berikutnya}} \\ \hline
  \endfoot

  % Last footer without next page indication
  \hline \hline
  \endlastfoot

 \hline
  1 & Selain Bupati Saiful Ilah, Ada Raja Koruptor yang Sedang diburu KPK! \\ 
   & \textit{http://fnn.co.id/post/selain-bupati-saiful-ilah-ada-raja-koruptor-yang-sedang-diburu-kpk} \\
 \hline
  2 & Ruslan Tawari: Pejabat Bupati Seram Barat Jangan Bikin Resah Rakyat \\
   & \textit{http://fnn.co.id/post/rustlam-tawari-pejabat-bupati-seram-barat-jangan-bikin-resah-rakyat-\#back-to-top} \\
 \hline
  3 & Ruslan Tawari: Pejabat Bupati Seram Barat Jangan Bikin Resah Rakyat \\
   & \textit{http://fnn.co.id/post/rustlam-tawari-pejabat-bupati-seram-barat-jangan-bikin-resah-rakyat-\#1} \\
 \hline
  4 & Ruslan Tawari: Pejabat Bupati Seram Barat Jangan Bikin Resah Rakyat \\
   & \textit{http://fnn.co.id/post/rustlam-tawari-pejabat-bupati-seram-barat-jangan-bikin-resah-rakyat-\#1} \\
 \hline
  5 & Ruslan Tawari: Pejabat Bupati Seram Barat Jangan Bikin Resah Rakyat \\
   & \textit{http://fnn.co.id/post/rustlam-tawari-pejabat-bupati-seram-barat-jangan-bikin-resah-rakyat-\#1} \\
 \hline
\end{longtable}
\end{center}

\begin{center}
  \begin{longtable}[c]{|p{0.2\textwidth}|p{0.8\textwidth}|} 
  \caption{\label{tabel:hasil_tfidf_bupati} Hasil pencarian dengan kata kunci
  \textit{bupati} dengan metode \textit{TF-IDF}} \\

  \hline
  % First page header
    \multicolumn{1}{|p{0.2\textwidth}|}{\textbf{Peringkat}} & \multicolumn{1}{|p{0.8\textwidth}|}{\textbf{Hasil}}\\ \hline 
  \endfirsthead

  % Next page header
  \multicolumn{2}{c}%
    {{\textbf{\tablename\ \thetable{}:} Hasil pencarian dengan kata kunci
    \textit{bupati} dengan metode \textit{TF-IDF}}} \\
  \hline
  \multicolumn{1}{|c|}{\textbf{Peringkat}} & \multicolumn{1}{|c|}{\textbf{Hasil}}\\ \hline 
  \endhead

  % Next page indication footer
  \hline \multicolumn{2}{|r|}{{Dilanjutkan pada halaman berikutnya}} \\ \hline
  \endfoot

  % Last footer without next page indication
  \hline \hline
  \endlastfoot

 \hline
    1 & \textit{http://fnn.co.id/post/jahiliyah-yang-belum-berevolusi} \\ 
 \hline
    2 & \textit{http://fnn.co.id/post/wakil-ketua-dpr-santri-harus-jadi-penopang-kekuatan-ekonomi-baru} \\
 \hline
    3 & \textit{http://fnn.co.id/post/pemprov-sulsel-resmikan-ruas-jalan-perbatasan-sidrap-soppeng} \\
 \hline
    4 & \textit{http://fnn.co.id/category/daerah?page5\#back-to-top} \\
 \hline
    5 & \textit{http://fnn.co.id/category/daerah?page=18\#3} \\
 \hline
\end{longtable}
\end{center}

% Bupati

% Berikut adalah hasil pencarian dengan kata kunci \textit{waskita}.

\begin{center}
\begin{longtable}{|c|c|} 
  \caption{\label{tabel:hasil_inv_waskita} Hasil pencarian dengan kata kunci
  \textit{waskita} dengan metode \textit{inverted index}} \\

  \hline
  % First page header
  \multicolumn{1}{|c|}{\textbf{Peringkat}} & \multicolumn{1}{|c|}{\textbf{Hasil}}\\ \hline 
  \endfirsthead

  % Next page header
  \multicolumn{2}{|c|}%
    {{\textbf{\tablename\ \thetable{}:} \textit{Inverted file index} dari lirik lagu pada tabel
    \ref{tab:lirik}}} \\
  \multicolumn{1}{|c|}{\textbf{Peringkat}} & \multicolumn{1}{|c|}{\textbf{Hasil}}\\ \hline 
  \endhead

  % Next page indication footer
  \hline \multicolumn{2}{r}{{Dilanjutkan pada halaman berikutnya}} \\ \hline
  \endfoot

  % Last footer without next page indication
  \hline \hline
  \endlastfoot
 % \hline
 %  Peringkat & Hasil \\ 
 \hline
  1 & Dimas Huda \\ 
   & \textit{http://fnn.co.id/author/Dimas Huda} \\
 \hline
  2 & infrastruktur \\
   & \textit{http://fnn.co.id/category/infrastruktur} \\
 \hline
  3 & hukum \\
   & \textit{http://fnn.co.id/category/hukum?page=4} \\
 \hline
  4 & Mochamad Toha \\
   & \textit{http://fnn.co.id/author/Mochamad Toha} \\
 \hline
  5 & editorial \\
   & \textit{http://fnn.co.id/category/editorial} \\
 \hline
\end{longtable}
\end{center}

\begin{table}[H]
\begin{center}
  \caption{\label{tabel:hasil_gst_waskita} Hasil pencarian dengan kata kunci
  \textit{waskita} dengan metode \textit{inverted index} dan integrasi dengan
  \textit{GST}}
  \begin{tabular}{|p{0.7in}|p{4.5in}|} 
 \hline
  Peringkat & Hasil \\ 
 \hline
  1 & Waskita Karya dan PT API Tandatangani Divestasi Tol Cibitung-Cilincing \\ 
   & \textit{http://fnn.co.id/post/waskita-karya-dan-pt-api-tandatangani-divestasi-tol-cibitung-cilincing} \\
 \hline
  2 & Sim Salabim Abracadabra Waskita Karya \\
   & \textit{http://fnn.co.id/post/simbsalabim-waskita-karya\#back-to-top} \\
 \hline
  3 & Sim Salabim Abracadabra Waskita Karya \\
   & \textit{http://fnn.co.id/post/simbsalabim-waskita-karya\#1} \\
 \hline
  4 & Sim Salabim Abracadabra Waskita Karya \\
   & \textit{http://fnn.co.id/post/simbsalabim-waskita-karya\#2} \\
 \hline
  5 & Sim Salabim Abracadabra Waskita Karya \\
   & \textit{http://fnn.co.id/post/simbsalabim-waskita-karya\#3} \\
 \hline
\end{tabular}
\end{center}
\end{table}

\begin{table}[H]
\begin{center}
  \caption{\label{tabel:hasil_tfidf_waskita} Hasil pencarian dengan kata kunci
  \textit{waskita} dengan metode \textit{TF-IDF}}
  \begin{tabular}{|p{0.7in}|p{4.5in}|} 
 \hline
  Peringkat & Hasil \\ 
 \hline
    1 & \textit{http://fnn.co.id/post/ptm-100-persen-di-surabaya-belum-bisa-dilaksanakan-secara-penuh\#carousel-example-generic} \\ 
 \hline
    2 & \textit{http://fnn.co.id/post/waskita-dan-pt-smi-teken-pjbb-divestasi-tol-cimanggis-cibitung\#back-to-top} \\
 \hline
    3 & \textit{http://fnn.co.id/post/waskita-dan-pt-smi-teken-pjbb-divestasi-tol-cimanggis-cibitung\#1} \\
 \hline
    4 & \textit{http://fnn.co.id/post/waskita-dan-pt-smi-teken-pjbb-divestasi-tol-cimanggis-cibitung\#2} \\
 \hline
    5 & \textit{http://fnn.co.id/post/waskita-dan-pt-smi-teken-pjbb-divestasi-tol-cimanggis-cibitung\#3} \\
 \hline
\end{tabular}
\end{center}
\end{table}

% Berikut adalah hasil pencarian dengan kata kunci \textit{nuklir}.

\begin{table}[H]
\begin{center}
  \caption{\label{tabel:hasil_inv_nuklir} Hasil pencarian dengan kata kunci
  \textit{nuklir} dengan metode \textit{inverted index}}
\begin{tabular}{|c|c|} 
 \hline
  Peringkat & Hasil \\ 
 \hline
  1 & Mochamad Toha \\ 
   & \textit{http://fnn.co.id/author/Mochamad Toha} \\
 \hline
  2 & energi \\
   & \textit{http://fnn.co.id/category/energi?page=2} \\
 \hline
  3 & energi \\
   & \textit{http://fnn.co.id/category/energi?page=7} \\
 \hline
  4 & Dimas Huda \\
   & \textit{http://fnn.co.id/author/Dimas Huda} \\
 \hline
  5 & nasional \\
   & \textit{http://fnn.co.id/category/nasional?page=5} \\
 \hline
\end{tabular}
\end{center}
\end{table}

\begin{table}[H]
\begin{center}
  \caption{\label{tabel:hasil_gst_nuklir} Hasil pencarian dengan kata kunci
  \textit{nuklir} dengan metode \textit{inverted index} dan integrasi dengan
  \textit{GST}}
  \begin{tabular}{|p{0.7in}|p{4.5in}|} 
 \hline
  Peringkat & Hasil \\ 
 \hline
  1 & Malaysia Prihatin Perlucutan Senjata Nuklir Melambat \\ 
   & \textit{http://fnn.co.id/post/malaysia-prihatin-perlucutan-senjata-nuklir-melambat} \\
 \hline
  2 & Penumpukan Nuklir China Mengkhawatirkan Amerika Serikat \\
   & \textit{http://fnn.co.id/post/penumpukan-nuklir-china-mengkhawatirkan-ameriak-serikat} \\
 \hline
  3 & Indonesia Minta Perjanjian Nonproliferasi Nuklir Ditegakkan \\
   & \textit{http://fnn.co.id/post/indonesia-minta-perjanjian-nonproliferasi-nuklir-ditegakkan} \\
 \hline
  4 & PLTN Ukraina Terbakar, Media Rusia Sebut Radiasi Nuklir Aman \\
   & \textit{http://fnn.co.id/post/pltn-ukraina-terbakar-media-rusia-sebut-radiasi-nuklir-aman\#back-to-top} \\
 \hline
  5 & PLTN Ukraina Terbakar, Media Rusia Sebut Radiasi Nuklir Aman \\
   & \textit{http://fnn.co.id/post/pltn-ukraina-terbakar-media-rusia-sebut-radiasi-nuklir-aman\#1} \\
 \hline
\end{tabular}
\end{center}
\end{table}

\begin{table}[H]
\begin{center}
  \caption{\label{tabel:hasil_tfidf_nuklir} Hasil pencarian dengan kata kunci
  \textit{nuklir} dengan metode \textit{TF-IDF}}
  \begin{tabular}{|p{0.7in}|p{4.5in}|} 
 \hline
  Peringkat & Hasil \\ 
 \hline
    1 & \textit{http://fnn.co.id/post/bsi-perluas-ekosistem-ekonomi-digital-lewat-dewan-masjid-indonesia} \\ 
 \hline
    2 & \textit{http://fnn.co.id/post/dpp-psi-tegaskan-viani-limardi-bukan-lagi-kader-partai} \\
 \hline
    3 & \textit{http://fnn.co.id/post/kementrian-investasi-permudah-kemitraan-umkm-usaha-besar} \\
 \hline
    4 & \textit{http://fnn.co.id/post/wapres-bersurat-ke-menkeu-dan-bpjph-guna-percepat-kodifikasi-halal} \\
 \hline
    5 & \textit{http://fnn.co.id/post/anies-baswedan-senang-bisa-bantu-tugas-kpk} \\
 \hline
\end{tabular}
\end{center}
\end{table}

% Berikut adalah hasil pencarian dengan kata kunci \textit{tiket coldplay}.

\begin{table}[H]
\begin{center}
  \caption{\label{tabel:hasil_inv_coldplay} Hasil pencarian dengan kata kunci
  \textit{tiket coldplay} dengan metode \textit{inverted index}}
\begin{tabular}{|c|c|} 
 \hline
  Peringkat & Hasil \\ 
 \hline
  1 & Mochammad Toha \\ 
   & \textit{http://fnn.co.id/author/Mochammad Toha} \\
 \hline
  2 & politik \\
   & \textit{http://fnn.co.id/category/politik?page=14} \\
 \hline
  3 & hukum \\
   & \textit{http://fnn.co.id/category/hukum?page=10} \\
 \hline
  4 & hukum \\
   & \textit{http://fnn.co.id/category/hukum?page=9} \\
 \hline
  5 & ekonomi \\
   & \textit{http://fnn.co.id/category/ekonomi?page=6} \\
 \hline
\end{tabular}
\end{center}
\end{table}

\begin{table}[H]
\begin{center}
  \caption{\label{tabel:hasil_gst_coldplay} Hasil pencarian dengan kata kunci
  \textit{tiket coldplay} dengan metode \textit{inverted index} dan integrasi dengan
  \textit{GST}}
  \begin{tabular}{|p{0.7in}|p{4.5in}|} 
 \hline
  Peringkat & Hasil \\ 
 \hline
  1 & Coldplay "LGBT" Kok Kagum Tokoh Syiah? \\ 
   & \textit{http://fnn.co.id/post/coldplay-lgbt-kok-kagum-tokoh-syiah} \\
 \hline
  2 & Coldplay "LGBT" Kok Kagum Tokoh Syiah? \\ 
   & \textit{http://fnn.co.id/post/coldplay-lgbt-kok-kagum-tokoh-syiah\#back-to-top} \\
 \hline
  3 & Coldplay "LGBT" Kok Kagum Tokoh Syiah? \\ 
   & \textit{http://fnn.co.id/post/coldplay-lgbt-kok-kagum-tokoh-syiah\#1} \\
 \hline
  4 & Coldplay "LGBT" Kok Kagum Tokoh Syiah? \\ 
   & \textit{http://fnn.co.id/post/coldplay-lgbt-kok-kagum-tokoh-syiah\#2} \\
 \hline
  5 & Coldplay "LGBT" Kok Kagum Tokoh Syiah? \\ 
   & \textit{http://fnn.co.id/post/coldplay-lgbt-kok-kagum-tokoh-syiah\#3} \\
 \hline
\end{tabular}
\end{center}
\end{table}

\begin{table}[H]
\begin{center}
  \caption{\label{tabel:hasil_tfidf_coldplay} Hasil pencarian dengan kata kunci
  \textit{tiket coldplay} dengan metode \textit{TF-IDF}}
  \begin{tabular}{|p{0.7in}|p{4.5in}|} 
 \hline
  Peringkat & Hasil \\ 
 \hline
    1 & \textit{http://fnn.co.id/post/era-jokowi-sudah-padam\#carousel-example-generic} \\ 
 \hline
    2 & \textit{http://fnn.co.id/post/bareskrim-mendalami-dugaan-penipuan-penjualan-tiket-online-coldplay\#back-to-top} \\
 \hline
    3 & \textit{http://fnn.co.id/post/bareskrim-mendalami-dugaan-penipuan-penjualan-tiket-online-coldplay\#1} \\
 \hline
    4 & \textit{http://fnn.co.id/post/bareskrim-mendalami-dugaan-penipuan-penjualan-tiket-online-coldplay\#2} \\
 \hline
    5 & \textit{http://fnn.co.id/post/bareskrim-mendalami-dugaan-penipuan-penjualan-tiket-online-coldplay\#3} \\
 \hline
\end{tabular}
\end{center}
\end{table}

% Berikut adalah hasil pencarian dengan kata kunci \textit{ganjar pranowo}.

\begin{table}[H]
\begin{center}
  \caption{\label{tabel:hasil_inv_ganjar} Hasil pencarian dengan kata kunci
  \textit{ganjar pranowo} dengan metode \textit{inverted index}}
\begin{tabular}{|c|c|} 
 \hline
  Peringkat & Hasil \\ 
 \hline
  1 & Mochammad Toha \\ 
   & \textit{http://fnn.co.id/author/Mochammad Toha} \\
 \hline
  2 & Ganjar Calon Presiden yang Paling Lemah \\
   & \textit{http://fnn.co.id/post/ganjar-calon-presiden-yang-paling-lemah} \\
 \hline
  3 & politik \\
   & \textit{http://fnn.co.id/category/politik?page=3} \\
 \hline
  4 & politik \\
   & \textit{http://fnn.co.id/category/politik?page=10} \\
 \hline
  5 & politik \\
   & \textit{http://fnn.co.id/category/politik?page=15} \\
 \hline
\end{tabular}
\end{center}
\end{table}

\begin{table}[H]
\begin{center}
  \caption{\label{tabel:hasil_gst_ganjar} Hasil pencarian dengan kata kunci
  \textit{ganjar pranowo} dengan metode \textit{inverted index} dan integrasi dengan
  \textit{GST}}
  \begin{tabular}{|p{0.7in}|p{4.5in}|} 
 \hline
  Peringkat & Hasil \\ 
 \hline
  1 & Ganjar Semakin "Bandel", dia Tahu Dilema Megawati \\ 
   & \textit{http://fnn.co.id/post/ganjar-semakin-bandel-dia-tahu-dilema-megawati} \\
 \hline
  2 & Tiga persoalan Serius Pasca Pencapresan Ganjar \\ 
   & \textit{http://fnn.co.id/post/tiga-persoalan-serius-pasca-pencapresan-ganjar} \\
 \hline
  3 & Megawati Akhirnya Akan Mendukung Ganjar Pranowo \\ 
   & \textit{http://fnn.co.id/post/megawati-akhirnya-akan-mendukung-ganjar-pranowo} \\
 \hline
  4 & Megawati Akhirnya Akan Mendukung Ganjar Pranowo \\ 
   & \textit{http://fnn.co.id/post/megawati-akhirnya-akan-mendukung-ganjar-pranowo\#back-to-top} \\
 \hline
  5 & Megawati Akhirnya Akan Mendukung Ganjar Pranowo \\ 
   & \textit{http://fnn.co.id/post/megawati-akhirnya-akan-mendukung-ganjar-pranowo\#1} \\
 \hline
\end{tabular}
\end{center}
\end{table}

\begin{table}[H]
\begin{center}
  \caption{\label{tabel:hasil_tfidf_ganjar} Hasil pencarian dengan kata kunci
  \textit{ganjar pranowo} dengan metode \textit{TF-IDF}}
  \begin{tabular}{|p{0.7in}|p{4.5in}|} 
 \hline
  Peringkat & Hasil \\ 
 \hline
    1 & \textit{http://fnn.co.id/post/yaqut-harus-kukut} \\ 
 \hline
    2 & \textit{http://fnn.co.id/post/dimulai-dari-palesrina-dunia-diambang-perang-agama} \\
 \hline
    3 & \textit{http://fnn.co.id/post/dpr-ri-usulkan-pemilu-serentak-6-maret-2024} \\
 \hline
    4 & \textit{http://fnn.co.id/post/presiden-main-sandiwara} \\
 \hline
    5 & \textit{http://fnn.co.id/post/ketimbang-melakukan-amandemen-mpr-diminta-fokus-sosialisasi-empat-pilar} \\
 \hline
\end{tabular}
\end{center}
\end{table}

\subsection{Pengujian Performa}

Berikut adalah hasil pengujian perbandingan performa antara penggunaan struktur 
index dengan integrasi \textit{GST}.

\begin{table}[H]
\begin{center}
  \caption{\label{tabel:perbandingan_performa} Perbandingan durasi waktu 
  pencarian}
\begin{tabular}{|c|c|c|c|} 
 \hline
  Percobaan & Inverted Index & Inverted Index + GST & Perubahan \\ 
 \hline
  \textit{nuklir} & 0.12565734s & 0.00722047s & +94.3\% \\ 
  \textit{waskita} & 0.07239101s & 0.00734384s & +89.8\% \\
  \textit{bupati} & 1.11867232s & 0.03204097s & +97.1\% \\
  \textit{ganjar pranowo} & 6.17200237s & 0.08626969s & +98.6\% \\ 
  \textit{tiket coldplay} & 0.25978417s & 0.00856094s & +96.7\% \\
 \hline
\end{tabular}
\end{center}
\end{table}

Apabila dilihat dari perbandingan waktu yang dihabiskan, maka waktu yang 
digunakan untuk mengolah \textit{query} yang diberikan oleh pengguna diproses 
lebih cepat dengan menggunakan integrasi \textit{GST}. Oleh karena itu, 
berdasarkan skenario pengujian yang direncanakan, dapat disimpulkan bahwa hasil 
rancangan sesuai atau berhasil.

\section{Analisis Hasil}

Berdasarkan data yang telah dikumpulkan, analisis hasil pengujian adalah sebagai 
berikut:

\begin{enumerate}
  \item{Pencarian berhasil dilakukan dalam waktu kurang dari satu detik untuk 
    semua kasus pengujian.}
  \item{Integrasi dengan struktur \textit{GST} memberikan peningkatan performa 
    waktu pencarian karena menghindari melakukan iterasi pada seluruh
    \textit{hitlist} dari sebuah kata. Karena struktur \textit{GST} dapat 
    langsung memberikan lokasi dokumen, maka hanya diperlukan perpotongan yang 
    sesuai.}
  \item{Karena struktur \textit{GST} membutuhkan pemetaan antara dokumen dengan 
    kata yang ada di dalamnya, maka diperlukan \textit{hitlist} yang sama 
    persis, dengan perbedaan hanya pada dengan apa data tersebut dipetakan. Hal 
    ini membuat penggunaan memori dan penyimpanan persisten naik dua kali lipat.}
  \item{Proses pembuatan struktur \textit{GST} menghabiskan mayoritas waktu pada 
    proses indeks ulang, bahkan lebih lama dari proses pembuatan struktur indeks 
    itu sendiri.}
  \item{Peningkatan performa dari integrasi \textit{GST} dapat berasal dari dua 
    hal, yaitu karena dokumen yang mengandung kata tersebut bisa langsung 
    ditemukan dan daftar \textit{hit} yang perlu dihitung lebih pendek karena 
    hanya menghitung kata per dokumen ketimbang seluruh kata yang terekam di 
    database.}
\end{enumerate}

Hasil pencarian dengan index menunjukkan relevansi yang kurang baik jika 
dibandingkan dengan hasil metode \textit{TF-IDF}. Walaupun hasil pencarian 
hampir tidak menunjukkan halaman dengan judul yang sesuai, tetapi beberapa 
menghasilkan halaman web yang memiliki kategori yang sesuai.

Ketika dilakukan perbandingan lebih lanjut, nilai dari \textit{exact match} 
tertimpa oleh banyaknya jumlah kata yang sesuai dengan input. Penggunaan filter 
\textit{outlier} tidak memberikan hasil yang diharapkan.

Tetapi jika dibandingkan dengan hasil pada integrasi GST, hasil yang diberikan 
lebih akurat. Jika dilihat dari sumber datanya, terdapat dua perbedaan yang 
cukup signifikan. Struktur \textit{inverted index} menggunakan informasi yang 
berasal hanya dari paragraf yang ada pada suatu halaman. Sementara struktur
\textit{GST} menggunakan informasi yang berasal dari judul halaman.

% Baris ini digunakan untuk membantu dalam melakukan sitasi
% Karena diapit dengan comment, maka baris ini akan diabaikan
% oleh compiler LaTeX.
\begin{comment}
\bibliography{daftar-pustaka}
\end{comment}

% %!TEX root = ./template-skripsi.tex
%-------------------------------------------------------------------------------
%                          BAB V
%               		KESIMPULAN DAN SARAN
%-------------------------------------------------------------------------------

\chapter{KESIMPULAN DAN SARAN}

\section{Kesimpulan}

Berdasarkan hasil dari implementasi dan pengujian terhadap arsitektur
\textit{inverted index}, maka diperoleh kesimpulan sebagai berikut:

\begin{enumerate}
	\item{Data pada kolom \textit{paragraph} dengan jumlah 1.010.083 baris dari
		tabel \textit{page\_paragraph} berhasil digunakan untuk membuat struktur
		\textit{inverted index}.}
	\item{Modul \textit{indexer} berbasis \textit{inverted index} yang digunakan
		untuk melakukan pemeringkatan pada \textit{query} yang diberikan oleh 
		pengguna telah selesai dibuat.}
	\item{Penyimpanan persisten untuk struktur \textit{inverted index} 
		diimplementasikan sebagai penyimpanan sederhana dalam satu proses yang sama 
		dengan modul \textit{indexer}.}
	\item{Integrasi dengan struktur \textit{GST} yang telah dibuat pada penelitian 
		sebelumnya berhasil diintegrasikan dengan struktur \textit{inverted index}.}
	\item{Integrasi dengan struktur \textit{GST} dapat memberikan reduksi waktu 
		pencarian yang signifikan jika dibandingkan hanya dengan penggunaan
		\textit{inverted index} saja, dengan rasio perbedaan berkisar antara 85 
		- 98\%.}
	\item{Relevansi pencarian tidak terlalu baik jika hanya menggunakan
		\textit{inverted index}. Hasil menjadi lebih relevan ketika menggunakan 
		integrasi \textit{GST}.}
	\item{Informasi yang ada pada \textit{tag} paragraf (\textit{<p>}) tidak cukup 
		untuk menjadi dasar dalam pemeringkatan hasil pencarian.}
\end{enumerate}

\section{Saran}

Adapun saran untuk penelitian selanjutnya adalah:
\begin{enumerate} 
	\item{Melanjutkan penelitian tentang informasi lain yang dapat di ekstrak dari 
		suatu halaman web untuk meningkatkan hasil pencarian.}
	\item{Melakukan perombakan terhadap struktur \textit{hitlists} agar bisa 
		direferensikan melalui alamat memori untuk menghilangkan redundansi pada 
		struktur dengan integrasi \textit{GST}.}
	\item{Melanjutkan penelitian tentang penggunaan kompresi pada struktur
		\textit{hit} untuk mereduksi penggunaan memori.}
	\item{Menuliskan implementasi \textit{indexer} dengan menggunakan bahasa lain 
		yang lebih cepat dan mampu mengeksploitasi sistem \textit{multi-thread} 
		dengan lebih baik seperti \textit{Rust} atau \textit{C++}.}
	\item{Melanjutkan penelitian tentang distribusi penyimpanan dan penggunaan 
		\textit{hitlists} pada memori.}
\end{enumerate}


% Baris ini digunakan untuk membantu dalam melakukan sitasi
% Karena diapit dengan comment, maka baris ini akan diabaikan
% oleh compiler LaTeX.
\begin{comment}
\bibliography{daftar-pustaka}
\end{comment}


%-----------------------------------------------------------------
%Disini akhir masukan Bab
%-----------------------------------------------------------------


%-----------------------------------------------------------------
% Disini awal masukan untuk Daftar Pustaka
% - Daftar pustaka diambil dari file .bib yang ada pada folder ini
%   juga.
% - Untuk memudahkan dalam memanajemen dan menggenerate file .bib
%   gunakan reference manager seperti Mendeley, Zotero, EndNote,
%   dll.
%-----------------------------------------------------------------
\bibliography{daftar-pustaka}
\bibliographystyle{apalike}
\addcontentsline{toc}{chapter}{DAFTAR PUSTAKA}
%-----------------------------------------------------------------
%Disini akhir masukan Daftar Pustaka
%-----------------------------------------------------------------


\end{document}
