%!TEX root = ./template-skripsi.tex
%-------------------------------------------------------------------------------
% 								BAB I
% 							LATAR BELAKANG
%-------------------------------------------------------------------------------

\chapter{PENDAHULUAN}

\section{Latar Belakang Masalah }

% Intro
Mesin pencari atau \emph{search engine} adalah program komputer yang digunakan 
untuk melakukan pencarian situs web. Mesin pencari membutuhkan data ketersediaan
situs web sebelum dapat digunakan.

% Sejarah mesin pencari
Dalam perkembangannya, berbagai metode digunakan untuk mengumpulkan dan
menyimpan data situs web yang tersedia. \emph{Archie}, mesin pencari pertama, 
mengumpulkan data dengan cara mencari daftar situs yang tersedia secara publik
menggunakan protokol \emph{Telnet} dan memperbaharui nya secara berkala.

% Arsitektur Google
Mesin pencari membutuhkan serangkaian proses yang perlu dilakukan sebelum dapat
menerima \emph{query} terkait informasi tertentu. Pada arsitektur \emph{Google},
terdapat beberapa komponen yang memiliki tugasnya tersendiri. Pertama,
diperlukan proses \emph{crawling} untuk mendapatkan kumpulan \emph{web page}.
Proses ini dilakukan oleh beberapa \emph{crawler}.

% Arsitektur Lazuardy
Upaya implementasi ulang arsitektur Google telah dilakukan oleh Lazuardy
Khatulistiwa dari Ilmu Komputer angkatan 18 Universitas Negeri Jakarta.
Arsitektur yang ada saat ini dimulai dari \emph{crawler} yang bertugas
mengumpulkan halaman web berdasarkan \emph{entrypoint} tertentu. Halaman web
tersebut kemudian di ekstrak dan dikumpulkan daftar kata yang termuat pada
halaman web tersebut serta \emph{outgoing link} yang merujuk kepada halaman web
lain. Kedua data tersebut kemudian disimpan pada database	\emph{MySQL}.

Data yang tersimpan pada database selanjutnya di proses oleh modul
\emph{PageRank}. Modul \emph{PageRank} menghitung peringkat suatu halaman web
berdasarkan seberapa banyak halaman web lain yang merujuk kepada halaman web
tersebut. Setelah kalkulasi peringkat halaman, data akan diolah oleh modul
\emph{TF-IDF}. Modul \emph{TF-IDF} akan menghitung bobot kata pada dokumen
berdasarkan frekuensi kemunculan kata tersebut dalam suatu dokumen tertentu dan
jumlah dokumen yang mengandung kata tersebut.

% Kekurangan dari arsitektur existing terutama pada bagian indexing
Hasil riset yang telah dilakukan oleh Lazuardy masih memiliki kekurangan, dan
implementasinya juga cukup berbeda dibandingkan dengan arsitektur milik Google.
Terdapat banyak komponen dari arsitektur Google yang belum diketahui detailnya,
salah satunya adalah terkait \emph{indexing}.

% Penjelasan terkait indexing
\emph{Indexing} adalah suatu proses pemetaan \emph{record} pada database dengan
tujuan mempercepat proses pengambilan \emph{record} dari database. Proses
\emph{indexing} akan menghasilkan \emph{index}, yaitu representasi data yang
merujuk pada lokasi data yang lebih lengkap pada database. Konsep \emph{index}
disini hampir sama dengan index yang biasa ditemukan di bagian belakang buku.

Representasi data oleh \emph{index} memiliki tingkat akurasi lokasi data
(\emph{granularity}) yang dapat diatur, seperti suatu frase dalam suatu
paragraf, atau unit data yang lebih kecil seperti satu kata tertentu saja.
Pemilihan tingkat \emph{granularity} dapat mempengaruhi performa dan akurasi
dari proses pengambilan data. Sebagai contoh, penggunaan \emph{synonym ring}
yang dapat melakukan pengelompokan kata yang bermakna sama seperti
\emph{WordNet} dapat memberikan hasil yang lebih baik dibandingkan dengan hanya
menggunakan potongan kata biasa~~\cite{gonzalo1998wordnet}.

% Prinsip kerja
Terdapat berbagai implementasi \emph{index} yang disesuaikan dengan jenis data
yang disimpan pada database. Pada kasus yang melibatkan teks, jenis yang paling
umum digunakan adalah \emph{inverted file index}. \emph{Inverted file index}
adalah daftar yang memuat kata dan posisinya dalam dokumen yang diurutkan
berdasarkan kata.  Untuk mendapatkan daftar kata pada dokumen, dibutuhkan
\emph{lexicon} yang memuat seluruh kata yang muncul pada database
~\cite{hersh2001gigabytes}.

% Google menggunakan algoritma indexing sendiri.
Saat ini implementasi ulang proses \emph{indexing} yang tepat untuk kebutuhan
mesin pencari sulit dilakukan karena detail dari arsitektur mesin pencari yang
bersifat tertutup. Google menggunakan implementasi \emph{filesystem} sendiri
yang sudah dioptimalkan baik performa maupun penggunaan \emph{storage}-nya
untuk kebutuhan mesin pencari.

% Jelaskan bahwa pada penelitian lazuardy baru PageRank dan TFIDF saja, sehingga nilai W nya belum jadi
% (masih ada beberapa metode yang perlu digabungkan untuk mendapatkan nilai dari W)
Arsitektur yang dibuat saat ini nantinya akan menggabungkan beberapa algoritma
pemeringkatan untuk mendapatkan skor \emph{query ranking}, dengan rumus:

\begin{equation}
	QR = \beta{}_1 \cdot{} QR_1 + \cdots{} + \beta{}_N \cdot{} QR_N
\end{equation}


% Tanyakan maksud dari indexing mengkapsulasi QR
% Indexing nantinya tetap akan berpengaruh kepada hasil ranking
% Rumuns pencarian dokumen
% Hasil W sudah ada
% Tutup penjelasan dengan menjelaskan langkah Google hingga sebelum proses
% ranking


% Crawler sangat dibutuhkan \emph{search engine} untuk dapat mencari informasi dengan lebih efisien. Proposal ini juga untuk memperdalam ilmu mengenai \emph{information retrieval} khususnya \emph{web crawling}. Oleh karena itu, perlu dibuat desain perancangan \emph{web crawler} pada \emph{search engine}. Dan ini tertuang pada penelitian yang berjudul \textbf{“Desain Perancangan \emph{Crawler} Sebagai Pendukung Pada \emph{Search Engine}”}.


\section{Rumusan Masalah}
Berdasarkan uraian pada latar belakang yang di utarakan di atas, maka perumusan
masalah pada penelitian ini adalah ``Bagaimana cara mendesain perancangan
\emph{crawler} sebagai pendukung pada \emph{search engine}?``

\section{Pembatasan Masalah}
Pembatasan masalah pada penelitian ini adalah pembuatan sebagian arsitektur
\emph{search engine} yaitu \emph{crawling}. Algoritma crawler yang akan dibuat
mengacu pada model algoritma \emph{Google} awal.

\section{Tujuan Penelitian}
\begin{enumerate}
	\item Memahami arsitektur mesin pencari.
	\item Memahami cara kerja proses \emph{indexing}.
	\item Membuat implementasi \emph{indexing} untuk memenuhi kebutuhan mesin
		pencari.
\end{enumerate}

\section{Manfaat Penelitian}
\begin{enumerate}
	\item Bagi penulis
		
	Menambah pengetahuan dibidang \emph{information retrieval} khususnya mengenai
		\emph{search engine} dan \emph{crawling}, mengasah kemampuan
		\emph{programming}, dan memperoleh gelar sarjana dibidang Ilmu Komputer.
		Selain itu, penulisan ini juga merupakan media bagi penulis untuk
		mengaplikasikan ilmu yang didapat di kampus ke kehidupan masyarakat.
		
	\item Bagi Universitas Negeri Jakarta 
	
	Menjadi pertimbangan dan evaluasi akademik khususnya Program Studi Ilmu
		Komputer dalam penyusunan skripsi sehingga dapat meningkatkan kualitas
		akademik di program studi Ilmu Komputer Universitas Negeri Jakarta serta
		meningkatkan kualitas lulusannya.
			
\end{enumerate}

% Baris ini digunakan untuk membantu dalam melakukan sitasi
% Karena diapit dengan comment, maka baris ini akan diabaikan
% oleh compiler LaTeX.
\begin{comment}
\bibliography{daftar-pustaka}
\end{comment}
