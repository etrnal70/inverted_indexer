%!TEX root = ./template-skripsi.tex
%-------------------------------------------------------------------------------
% 								BAB I
% 							LATAR BELAKANG
%-------------------------------------------------------------------------------

\chapter{PENDAHULUAN}

\section{Latar Belakang Masalah}

Mesin pencari atau \emph{search engine} adalah program komputer yang dirancang untuk melakukan pencarian situs web. \emph{Search engine} pertama kali diperkenalkan oleh Alan Emtage pada tahun 1990 dengan nama \emph{Archie}. Prinsip keja \emph{Archie} adalah melakukan pengindeksan semua \emph{file} pada web. Tujuannya untuk mempermudah pengguna internet mencari lokasi \emph{file} yang diinginkan~\cite{seymour2011history}.

% Intro

% Arsitektur Google
Mesin pencari membutuhkan serangkaian proses yang perlu dilakukan sebelum dapat menerima \emph{query} terkait informasi tertentu. Pada arsitektur \emph{Google},   terdapat beberapa komponen yang memiliki tugasnya tersendiri. Pertama, diperlukan proses \emph{crawling} untuk mendapatkan kumpulan \emph{web page}. Proses ini dilakukan oleh beberapa \emph{crawler} 

% Arsitektur Lazuardy
Upaya implementasi dari arsitektur Google telah dilakukan oleh Lazuardy Khatulistiwa dari Ilmu Komputer angkatan 18 Universitas Negeri Jakarta. Arsitektur yang sudah dibuat saat ini terdiri dari \emph{crawler} yang mencari data dari internet, \emph{MySQL} sebagai tempat penyimpanan, \emph{PageRank} sebagai metode skoring URL yang telah terkumpul, dan \emph{TF-IDF} sebagai metode skoring per kata.

% Kekurangan dari arsitektur existing terutama pada bagian indexing
Hasil riset yang telah dilakukan oleh Lazuardy masih memiliki banyak kekurangan, dan implementasinya juga cukup berbeda dibandingkan dengan arsitektur milik Google. Terdapat banyak komponen dari arsitektur Google yang belum diketahui detailnya, salah satunya adalah terkait \emph{indexing}.

% Google menggunakan algoritma indexing sendiri.

% Jelaskan bahwa pada penelitian lazuardy baru PageRank dan TFIDF saja, sehingga nilai W nya belum jadi
% (masih ada beberapa metode yang perlu digabungkan untuk mendapatkan nilai dari W)
%
% Tanyakan maksud dari indexing mengkapsulasi QR
% Rumuns pencarian dokumen
%

\emph{Archie} hanya menyediakan nama situs saja dalam daftar yang telah dibuat, sedangkan kontennya tidak ada. Pada tahun berikutnya mulai bermunculan mesin pencari baru. \emph{Aliweb} muncul pada tahun 1993, mesin pencari ini memberi kesempatan penggunanya untuk mengunggah halaman situs yang ingin terindeks di dunia maya, lengkap dengan deskripsi dari halaman tersebut. Kemudian, muncul \emph{Altavista} sebagai salah satu raksasa internet kala itu. \emph{Altavista} menyediakan fasilitas \emph{unlimited bandwith}, teknik dan sistem algoritmanya juga sudah lebih maju. Sembilan tahun setelahnya, muncul \emph{Yahoo}. \emph{Yahoo} menyediakan data dari berbagai situs yang sudah banyak dikenal orang. Para pemilik dan pengelola situs dapat menambahkan informasi yang dimiliki secara gratis, tetapi untuk fitur yang lengkap ada biaya yang harus dibayar. Perusahaan Yahoo kurang aktif dalam mengembangkan mesin pencarian mereka dan kurang bisa memuaskan para pengguna internet~\cite{seymour2011history}.

Mesin pencari yang saat ini masih populer yaitu \emph{Google}. \emph{Google} diluncurkan pada tahun 1998, sistem yang dikembangkan \emph{Google} relatif berbeda dengan para pendahulunya. \emph{Google} menggunakan sistem \emph{BackRub}, sistem ini mengandalkan \emph{backlink} sebagai alat pemeringkat. \emph{Google} membuat \emph{ranking} setiap halaman situs berdasarkan seberapa banyak penyebutan situs tersebut di situs-situs lain~\cite{brin1998can}. Selain \emph{Google}, mesin pencari yang masih populer saat ini yaitu \emph{Bing}. \emph{Bing} muncul di pertengahan 2009. Di Amerika Serikat, \emph{Bing} termasuk mesin pencari yang disukai karena memang ada sejumlah fitur yang cocok bagi masyarakat Amerika. Namun, di luar Amerika, \emph{Bing} kurang populer dan masih kalah dari \emph{Google}~\cite{seymour2011history}.

\begin{table}[H]
	\centering{}
	\caption{Penggunaan \emph{global search engine} Augustus 2020~\cite{netmarketshare}}\label{searchengine_net}
	\begin{tabular}{c c}
		\toprule
		\textbf{Search engine} & \textbf{Global share (\%)} \\
		\midrule
		Google & 83.46 \\
		\midrule
		Baidu & 7.35 \\
		\midrule
		Bing & 6.15 \\
		\midrule
		Yahoo! & 1.42 \\
		\midrule
		Yandex & 0.87 \\
		\midrule
		DuckDuckGo & 0.31 \\
		\bottomrule
	\end{tabular}
\end{table}

\emph{Apple} adalah perusahaan yang saat ini menggunakan \emph{Google} sebagai mesin pencarinya. Selama beberapa tahun ini \emph{Google} membayar \emph{Apple} lebih dari satu miliar dolar untuk tetap menjadi mesin pencari utama di \emph{safari} untuk \emph{iOS}, \emph{iPadOS}, dan \emph{macOS}. \emph{Apple} sudah membuat \emph{web crawler} bernama \emph{Applebot} untuk \emph{search engine Apple} dan kemungkinan \emph{search engine} tersebut akan rilis di tahun ini~\cite{jonhenshaw_2020}.

Fungsi dari mesin pencari selain mencari informasi adalah sebagai media pemasaran. Bertambahnya jumlah pengguna setiap detiknya membuat mesin pencari menjadi media pemasaran modern saat ini. Pengguna dapat mencari apa saja melalui mesin pencari, termasuk barang-barang yang ingin mereka beli. Mesin pencari membuat pencarian informasi semakin mudah. Pengguna yang sangat banyak membuat mesin pencari menjadi peluang tersendiri untuk memasarkan produk atau jasa bagi pemilik usaha atau perusahaan.

\emph{Google} memasangkan iklan pada mesin pencarinya untuk mendapatkan keuntungan, layanan iklan ini bernama \emph{Adwords}. \emph{Adwords} memungkinkan pengiklan menjangkau pengguna \emph{online} dengan beriklan di platform \emph{Google}. Jika mencari sebuah informasi suatu barang, maka beberapa baris hasil penelusuran teratas yang disematkan dengan tulisan “Ad” akan terlihat. Informasi tersebut dipasang oleh pengiklan yang menggunakan fitur \emph{Adword Google}.

\emph{Google} pun juga mendapatkan banyak data melalui mesin pencariannya. Selain itu, \emph{Google} juga membuat layanan lain seperti \emph{Gmail}, \emph{YouTube}, \emph{Search}, \emph{Drive}, \emph{Maps}, dan \emph{PlayStore}. Setiap layanan memiliki fungsi tersendiri untuk membantu aktivitas masyarakat. Dan dari setiap layanan \emph{Google} terdapat data-data pengguna yang dapat dipakai untuk keperluan lainnya.

%masukan kelebihan kekurangan web crawler sebelumnya

\emph{Web crawler} merupakan \emph{software} yang bekerja otomatis untuk menjelajahi \emph{World Wide Web} secara terorganisir. Kaur\cite{kaur2020simhar} dalam jurnalnya, telah membuat \emph{web crawler} dengan menggunakan \emph{focused crawler}. \emph{Focused web crawler} dapat memberikan hasil halaman relevan yang maksimum. Komponen pada \emph{focused web crawler} hampir sama dengan \emph{web crawler} pada umumnya, yaitu:

%membahas kelebihan dan kekurangan crawler SIMHAR

Proposal ini dikerjakan bersama dengan Savira Rahmayanti dari Ilmu Komputer angkatan 2015 Universitas Negeri Jakarta, akan mengembangkan \emph{server based search engine} yang bersifat \emph{open source} dan menjelaskan seperti apa arsitektur mesin pencari tersebut. Penulis mendapatkan bagian untuk mengimplementasi \emph{web crawling} pada \emph{search engine}, sedangkan Savira fokus untuk mengerjakan bagian \emph{searching} dan \emph{indexing}.

Pada proposal ini tidak mengimplementasikan apa yang telah dikembangkan oleh Sawroop Kaur~\cite{kaur2020simhar}, karena \emph{web crawler} tersebut sudah terlalu rumit. Tetapi proposal ini akan mengembangkan \emph{web crawler} dasar yang merujuk pada awal perkembangan \emph{search engine Google}~\cite{brin1998anatomy} dan desain web crawler Carlos Castillo~\cite{castillo2005effective}. Penlitian ini akan menjadi dasar pada penelitian \emph{search engine} berikutnya.

Crawler sangat dibutuhkan \emph{search engine} untuk dapat mencari informasi dengan lebih efisien. Proposal ini juga untuk memperdalam ilmu mengenai \emph{information retrieval} khususnya \emph{web crawling}. Oleh karena itu, perlu dibuat desain perancangan \emph{web crawler} pada \emph{search engine}. Dan ini tertuang pada penelitian yang berjudul \textbf{“Desain Perancangan \emph{Crawler} Sebagai Pendukung Pada \emph{Search Engine}”}.




\section{Rumusan Masalah}
Berdasarkan uraian pada latar belakang yang di utarakan di atas, maka perumusan masalah pada penelitian ini adalah “Bagaimana cara mendesain perancangan \emph{crawler} sebagai pendukung pada \emph{search engine}?”

\section{Pembatasan Masalah}
Pembatasan masalah pada penelitian ini adalah pembuatan sebagian arsitektur \emph{search engine} yaitu \emph{crawling}. Algoritma crawler yang akan dibuat mengacu pada model algoritma \emph{Google} awal.

\section{Tujuan Penelitian}
\begin{enumerate}
	\item Membuat \emph{crawler} yang dipakai untuk kebutuhan \emph{search engine}.
	\item Untuk mengetahui arsitektur \emph{search engine}.
	\item Untuk mempelajari cara kerja \emph{crawling}.
\end{enumerate}

\section{Manfaat Penelitian}
\begin{enumerate}
	\item Bagi penulis
		
	Menambah pengetahuan dibidang \emph{information retrieval} khususnya mengenai \emph{search engine} dan \emph{crawling}, mengasah kemampuan \emph{programming}, dan memperoleh gelar sarjana dibidang Ilmu Komputer. Selain itu, penulisan ini juga merupakan media bagi penulis untuk mengaplikasikan ilmu yang didapat di kampus ke kehidupan masyarakat.
		
	\item Bagi Universitas Negeri Jakarta 
	 	
	Menjadi pertimbangan dan evaluasi akademik khususnya Program Studi Ilmu Komputer dalam penyusunan skripsi sehingga dapat meningkatkan kualitas akademik di program studi Ilmu Komputer Universitas Negeri Jakarta serta meningkatkan kualitas lulusannya.
	 			
\end{enumerate}


% Baris ini digunakan untuk membantu dalam melakukan sitasi
% Karena diapit dengan comment, maka baris ini akan diabaikan
% oleh compiler LaTeX.
\begin{comment}
\bibliography{daftar-pustaka}
\end{comment}
